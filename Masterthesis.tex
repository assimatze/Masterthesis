\documentclass[a4paper,12pt]{scrartcl}
\usepackage[utf8]{inputenc}
\usepackage{lmodern}
\usepackage[ngerman]{babel}
\usepackage[autostyle=true,german=quotes]{csquotes}
\usepackage[T1]{fontenc}
\usepackage[dvips, final]{graphicx}
\usepackage{geometry}
\usepackage{setspace}
\usepackage{amsmath}
\usepackage{colortbl}
\usepackage{multirow}
\usepackage{afterpage}
\usepackage{wrapfig}
\usepackage{fancyhdr}
\usepackage{ae}

\usepackage[colorlinks=true,urlcolor=blue,linkcolor=black,
citecolor=black
]{hyperref}

\bibliographystyle{unsrt}

\geometry{verbose,a4paper,tmargin=30mm,bmargin=35mm,lmargin=25mm,rmargin=25mm}
\pagestyle{fancy}
\renewcommand{\sectionmark}[1]{\markboth{#1}{#1}}
\renewcommand{\subsectionmark}[1]{\markright{#1}}
\fancyhf{}
\lhead{\slshape \leftmark}
\rhead{\thepage}
\title{Deckblatt}
\author{Matthias Misior}

\begin{document}


\thispagestyle{empty}
\section*{Erklärung}
Hiermit versichere ich, dass ich die vorliegende Arbeit selbstständig verfasst und keine anderen als die angegebenen Quellen und Hilfsmittel benutzt habe sowie wörtliche und sinngemäße Zitate als solche gekennzeichnet habe. Diese Versicherung bezieht sich sowohl auf
Textinhalte sowie alle enthaltenen Abbildungen, Skizzen und Tabellen. Die Arbeit wurde in gleicher oder ähnlicher Form keiner anderen Prüfungsbehörde zur Erlangung eines akademischen Grades vorgelegt.
\\\\
Karlsruhe, den 04. April 2018
\\\\
Matthias Misior \noindent\dotfill

\newpage
\thispagestyle{empty}
\section*{Anerkennungen}
Zunächst möchte ich mich an dieser Stelle bei meinem Betreuer Maximilian Liesegang bedanken und meine Anerkennung ihm gegenüber zum Ausdruck bringen. Seine Fachkompetenz und sein Enthusiasmus haben mich über die gesamte Projektdauer begleitet und waren stets eine große Hilfe und eine wichtige Motivationsquelle für mich. 
\\\\
Des weiteren gilt mein persönlicher dank Frau Professorin Laubenheimer, dessen ständige Erreichbarkeit und schneller E-Mail Verkehr es mir erst ermöglicht haben, diese Arbeit in dieser Form rechtzeitig zum Abgabetermin fertigzustellen. 
\\\\
Zu Guter Letzt gilt mein persönlicher Dank dem gesamten esentri-Team. Das Team hat mich inspiriert, andere Wege zu gehen und ich habe dadurch viel über meine Vorgehensweise sowie über mich selbst gelernt. Die Fachkompetenz und Ratschläge des Teams waren dabei stets eine große Hilfe. Ohne die Unterstützung des esentri-Teams wäre diese Arbeit so nicht möglich gewesen.
\\\\
Ich hoffe ich konnte mit diesen einfachen Worten allen Beteiligten meine Dankbarkeit zum Ausdruck bringen. 

\newpage
\thispagestyle{empty}
\section*{Zusammenfassung}
Die esentri AG mit Sitz in der Technologieregion Karlsruhe ist ein mittelständisches Beratungs-Unternehmen mit den Schwerpunkten Digitalisierung und IT-Systemintegration. 
Das Unternehmen plant dabei seine Projekte agil, transparent und nah beim Kunden.
Um den hohen Ansprüchen dieser Kunden nun gerecht zu werden hat es sich die esentri zur Aufgabe gemacht, den internen Dokumentationsprozess innerhalb der Firma zu verbessern. 
\\\\
Diese Aufgabe erweist sich allerdings Aufgrund der in diesem Feld auftretenden Probleme als schwer greifbar. Das Schreiben von Dokumentationen wird meistens als langweilig und monoton angesehen. Hinzukommt dass das Dokumentieren nichts zum Lösungsansatz eines Projektes beiträgt. Aus diesen und weiteren Gründen wird das verfassen von Dokumentationen von vielen Entwicklern als “Zeitverschwendung” betrachtet. 
\\\\
Um diese Tätigkeit nun für die Mitarbeiter attraktiver zu gestalten und somit die Qualität der Dokumentationen zu verbessern, hat man sich dazu entschieden den Dokumentationsprozess mit Gamification aufzuwerten. Dafür soll sich in die Grundlagen des Gebietes Gamification eingearbeitet werden um zu ermitteln, ob und wie man die aus dem Gamification Bereich stammenden Elemente nutzen kann.


\section*{Abstract}
The esentri AG with their headquarters based in the technology region Karlsruhe is an medium-sized consulting-company with a focus on digitization and IT system integration. The company plans its projects agile, transparently and close to the customer. In order to meet the high demands of these customers, esentri has set itself the task of improving the internal documentation process within the company. 
\\\\
However, this task proves to be difficult to grasp due to the problems encountered in this field. Writing documentation is often considered boring and monotonous. In addition, documenting does not contribute to the solution approach of a project. For these and other reasons, writing documentation is considered a \enquote{waste of time} by many developers. 
\\\\
In order to make this activity more attractive for employees and thus improve the quality of documentation, it was decided to enhance the documentation process with gamification. For this purpose, the basics of gamification have to be studied in order to determine whether and how the elements from gamification can be used.     
\newpage
\thispagestyle{empty}
\tableofcontents
\thispagestyle{empty}
\newpage
\bibliographystyle{unsrt}


\section{Einleitung}
Dieses Kapitel dient als Einstieg in diese Arbeit und soll die Aufgaben, Ziele, Vorgehensweise und das Arbeitsumfeld genauer beschreiben. Dabei wird zwischen Aufgabenstellung und Aufgabenbearbeitung unterschieden. Die Aufgabenstellung lässt sich aus der Thesis Beschreibung ableiten und beschreibt die Aufgaben die im laufe dieser Thesis bearbeitet werden sollen. Die Aufgabenbearbeitung enthält die generelle Vorgehensweise wie diese Aufgaben umgesetzt werden sollen. Des weiteren werden die Ziele definiert die mit dieser Arbeit erreicht werden sollen. Die Arbeitsumgebung wird ebenfalls beschrieben und es wird am Ende dieses Kapitels mit einem Überblick abgeschlossen. 
\subsection{Aufgabenstellung}
Die Aufgabe die in dieser Arbeit behandelt werden soll ist, die Tätigkeit des Dokumentierens von Software (dokumentieren, Dokumentationsprozess) für die Softwareentwickler der esentri AG motivierender und belohnender zu gestalten. Hierfür soll der Dokumentationsprozess mit Spiele-Prinzipien und Spiele-Mechaniken versehen werden. Durch die Integration von Spiele-Prinzipien und Spiele-Mechaniken sollen dem Dokumentationsverfasser zusätzliche Anreize gegeben werden, gute und aussagekräftige Dokumentationen zu schreiben. 
\\\\
Das Integrieren von Spiele-Mechaniken und Spiele-Prinzipien in nicht Spiel-Kontexten, wie in diesem konkreten Fall den Dokumentationsprozess, wird als Gamification bezeichnet.
\\\\
Für die Integration müssen die verschiedensten Spiele-Prinzipien und Spiele-Mechaniken welche als Gamification Elemente zu verstehen sind, untersucht und bewertet werden. Dafür werden im Vorfeld Kriterien festgelegt anhand derer die verschiedenen Gamification Elemente analysiert werden können. Anschließend sollen die ermittelten Gamification Elemente in das Dokumentationstool der esentri AG integriert werden. 

\subsubsection{Definition Gamification}
Der Begriff Gamification wurde im Jahr 2002 von Nick Pelling geprägt, fand zu dieser Zeit aber keine Aufmerksamkeit und hatte auch mit der heutzutage geläufigen Definition wenig zu tun.
\\\\
\enquote{Gamification is the use of game mechanics and experience design to digitally engage and motivate people to achieve their goals.}\cite{gamificationDefinition}
\\\\
Erst 2009 wurde die Idee um Gamification wieder aufgegriffen und fand mit der digitalen Medienbranche auch ein vielseitiges Einsatzgebiet. Seit dieser Zeit (2009) sind mehrere verschiedene Geschäftsfelder um die Kernidee der Gamification entstanden wie z.B der Bereich der \enquote{Serious Games} oder auch das \enquote{e-learning} Umfeld. Einen detailierter Einblick in die verschiedene Bereiche der Gamification folgt in weiteren Kapiteln. 

\subsubsection{Nutzen von Gamification}
Dieses Zitat aus dem Film Mary Poppins beschreibt ziemlich präzise die Idee und den Nutzen, die hinter Gamification stecken:\\\\ 
\enquote{In every job that must be done, there is an element of fun. You find the fun, and - SNAP - the job's a game!} \footnotemark
\footnotetext{Film Mary Poppins, 1964} 
\\\\
Es geht darum, langweilige und monotone Tätigkeiten durch Gamification Elemente motivierender und interessanter zu gestalten. Dadurch soll zum einen der Anwender inspiriert werden. Zum anderen sollen die Resultate der Tätigkeit durch engagierte Anwender verbessert werden.

\subsection{Ziel dieser Arbeit} 
Ziel dieser Arbeit soll es sein, die Dokumentationsbereitschaft von Mitarbeitern der esentri AG zu erhöhen. Als Dokumentationsbereitschaft soll der Wille und die Fähigkeit von Mitarbeitern verstanden werden, gute und aussagekräftige Dokumentationen zu schreiben. Zu diesem Zweck sollen aus dem Bereich der Gamification diejenigen Elemente recherchiert werden, die diesem Ziel am förderlichsten sind. Die Gamification Elemente sollen dann wiederum in das bestehende Dokumentationstool Confluence integriert werden.
\begin{description}
   \item[Dokumentationsbereitschaft erhöhen bedeutet:]~\par
   \begin{itemize}
      \item Den \textbf{Willen/Motivation} der Mitarbeiter erhöhen, gute und aussagekräftige Dokumentationen zu schreiben.  
      \item Die \textbf{Fähigkeiten} der Mitarbeiter verbessern, gute und aussagekräftige Dokumentationen zu schreiben.
   \end{itemize}
\end{description}

\begin{description}
   \item[Durch das beeinflussen dieser beiden Punkte sollte es also möglich sein:]~\par
   \begin{itemize}
      \item Das Schreiben von Dokumentationen für den Schreiber interessanter zu gestalten.  
      \item Die Qualität der Dokumentationen zu erhöhen.
   \end{itemize}
\end{description}
Ein weiteres Ziel dieser Arbeit ist es, das Themengebiet rund um \enquote{Gamification} der esentri AG näher zu bringen. Der esentri AG ist es ein Anliegen, mehr über die potenziellen Einsatzmöglichkeiten dieses Themengebietes zu erfahren. Die Integration von Gamification Elementen in die vorhandenen Dokumentationstools der esentri AG sind somit als eine Art \enquote{Experiment} zu verstehen. Diese Arbeit soll anhand des Beispiels des Dokumentationstools aufzeigen, ob es sich rentiert Gamification Elemente in bestehende und zukünftige Projekte zu integrieren.  
\subsection{Arbeitsumgebung}
Die esentri AG mit Sitz in Ettlingen ist eine im Jahr 2002 gegründete Beratungs-Firma, die sich auf die beiden Geschäftsbereiche Digitalisierung und IT-Systemintegration spezialisiert hat. Die esentri ist ein cloudbasiertes innovatives Unternehmen das von sich selbst behauptet:
\\\\
\enquote{Unser Antrieb ist die Neugier, mit der wir nach den aktuellsten Trends Ausschau halten und mit der wir gerne auch mal über den Tellerrand schauen.} \footnotemark
\footnotetext{\url{http://www.esentri.com/unternehmen/}}
\\\\
Als solches bearbeitet die esentri neben Ihren Geschäften auch verschiedenste interne Projekte die sich mit neuen Technologien, Formaten oder Konzepten auseinandersetzen. Zu diesen Projekten zählt auch diese Arbeit. 
\\\\
Das Unternehmen verwaltet und plant seine Projekte agil und greift hierfür auf die Produktpalette von \textit{Atlassian} zurück. Zu den wichtigsten Atlassian Tools innerhalb esentri zählen unter anderem die Aufgabenmanagementsoftware \textit{Jira} und die \textit{Confluence Wiki}.

\subsubsection{Confluence}
Das Confluence Wiki wird seit 2012 bei der esentri eingesetzt. Wichtige Eigenschaften dieses Tools beinhalten die Möglichkeit, Verknüpfungen zwischen Jira, Bitbucket und weiteren Atlassians Produkten zu erzeugen. Somit ist es einfach möglich, Aufgaben Tickets von Jira mit einer speziellen Confluence Wikipage zu verknüpfen. Das stellt einen großen Vorteil gegenüber anderen Wikitools dar.
\\\\
\enquote{Confluence ist eine kommerzielle Wiki-Software, die vom australischen Unternehmen Atlassian entwickelt und als Enterprise Wiki hauptsächlich für die Kommunikation und den Wissensaustausch in Unternehmen und Organisationen verwendet wird, aber zunehmend auch als Basis für öffentliche Wikis im Internet zum Einsatz kommt.} \footnotemark
\footnotetext{\url{https://de.wikipedia.org/wiki/Confluence_(Atlassian)}}
\\\\
Die Atlassian Produkte verfügen zusätzlich über weitere \textit{Addons} und über einen eigenen Marketplace. Somit bleiben die Produkte erweiterbar und können individuell zugeschnitten werden.

\subsection{Aufgabenbearbeitung} 
Um die besagten Ziele zu erreichen, ist es wichtig die Kernaspekte dieser Arbeit zu verstehen, sie zu betrachten und ihre zusammenhänge aufzuführen.
\begin{description}
   \item[Die drei wichtigsten Aspekte dieser Arbeit lassen sich wie folgt einteilen:]~\par
   \begin{itemize}
      \item Dokumentation
      \item Gamification
      \item Confluence
   \end{itemize}
\end{description}
Diese drei Bereiche müssen behandelt werden um die im Vorherigen Abschnitt erwähnten Ziele, wie das erhöhen der Dokumentationsbereitschaft oder das aneignen von Grundlagen Wissen im Bereich Gamification zu erreichen.
\begin{description}
   \item[Dokumentation]~\par
Unter diesem Aspekt müssen die grundlegenden Fragen beantwortet werden wie:
   \begin{itemize}
      \item Wie schreibt man eine gute und aussagekräftige Dokumentation?
      \item Was sind Merkmale einer solchen Dokumentation?
      \item Warum ist das Schreiben von Dokumentationen eine eintönige Tätigkeit?
   \end{itemize}
\end{description}
Diese Fragen müssen als aller erstes untersucht und geklärt werden um die Kriterien und Anforderungen an den Bereich der Gamification aufstellen zu können. Diese Kriterien können dann verwendet werden um den Bereich der Gamification besser einzugrenzen. Dadurch wird das Risiko minimiert, sich im Gamification Bereich zu “verlieren”. 
\\\\
\textbf{Gamification}\\
Die Kriterien und Anforderungen welche sich aus der Analyse des Dokumentationsprozesses ergeben, sollen nun genutzt werden um die entsprechenden Gamification Elemente herauszuarbeiten. Es wird also untersucht, ob es Elemente im Bereich der Gamification gibt welche auf die Anforderungen und Kriterien passen.
\\\\
Um die Gamification Elemente und die Anforderungen besser aufeinander abstimmen zu können, ist es wichtig sich ebenfalls die Grundlagen zum Thema Gamification anzueignen. Dabei ist zu beachten, dass sich die Gamification Grundlagen nicht nur in technische Lösungen einteilen lassen. Gamification behandelt die Psychologie der Motivation und das erzeugen von Verhalten. In diesem Fall wollen wir die Gamification nutzen um Mitarbeiter zu motivieren und ihr Verhalten soweit ändern, das sie bessere Dokumentationen schreiben.
\begin{description}
   \item[Also müssen folgende Forschungsfragen in diesem Abschnitt behandelt werden:]~\par
   \begin{itemize}
      \item Was ist Motivation?
      \item Wie kann man Motivation erzeugen?
      \item Wie kann Motivation genutzt werden?
   \end{itemize}
\end{description}
Wenn diese Fragen geklärt sind, ist es möglich Aussagen darüber zu treffen, welche Gamification Elemente sich in Verbindung mit den Anforderungen am besten eignen würden um die Dokumentationsbereitschaft zu erhöhen.
\\\\
\textbf{Confluence}\\
Die identifizierten Gamification Elemente müssen zu guter Letzt noch in das firmeninterne Dokumentationstool \enquote{Confluence} integriert werden. Die benötigten Kenntnisse wie Confluence zu konfigurieren ist, welche Möglichkeiten der Konfiguration Confluence zu Verfügung stellt und wie aufwendig die Integration der gewählten Gamification Elemente wird, fallen in diesen Teil der Arbeit. 
\\\\
Die Confluence Software ist ein wichtiger Baustein der esentri AG und kann deswegen nicht ohne größeren Aufwand ausgetauscht werden. Deswegen ist es wichtig die Grenzen der Software zu ermitteln. Dadurch ist die Auswahl der Gamification Elemente zwar eingeschränkt, da lediglich Elemente verwendet werden dürfen die sich mit dieser Software auch umsetzen lassen. Trotzdem gibt es Punkte die dafür sprechen Confluence als Dokumentationstool für diese Arbeit zu wählen. So kann der Atlassian Marketplace z.B nach vorhandenen Gamification-Addons durchsucht werden die bereits Gamification Elemente beinhalten.
\\\\
\textbf{Zusammenhang von Dokumentation, Gamification und Confluence}\\
In dieser Arbeit wird versucht alle Schwerpunkte der einzelnen Themenbereiche Dokumentation, Gamification und Confluence zu betrachten und diese zusammenzuführen. Dadurch entstehen natürlich Abhängigkeiten und Zusammenhänge. So können z.B die Gamification Elemente erst ausgewählt werden nachdem die Anforderungen an die Elemente aus dem Dokumentationsprozess ermittelt wurden. Ebenfalls können diese Elemente nur dann verwendet werden wenn sie sich mit Confluence umsetzen lassen. Durch diese Abhängigkeiten aller drei Bereiche unterscheidet sich diese Arbeit von anderen Arbeiten, die eben nur eines dieser Felder als Schwerpunkt hat.
\subsection{Übersicht}
Diese Arbeit teilt sich in drei Teile auf. Im ersten Teil werden hauptsächlich die Grundlagen der wichtigen Aspekte Dokumentation, Gamification und Confluence behandelt. Es sollen Begrifflichkeiten und Definitionen dieser Bereiche aufgeführt und erläutert werden. 
\begin{description}
   \item[Aus dieser Phase werden die folgenden Teilergebnisse benötigt:]~\par
   \begin{itemize}
      \item Aus dem Bereich Dokumentation sollen die Anforderungen und Kriterien für die Gamification Elemente ermittelt werden.
      \item Die Anforderungen werden nun verwendet um aus dem Bereich der Gamification die passenden Elemente zu identifizieren.
      \item Im Confluence Teil muss geprüft werden, ob sich die identifizierten Gamification Elemente auch in Confluence umsetzen lassen. Andernfalls muss nach alternativen Lösungsansätzen geforscht werden.
   \end{itemize}
\end{description}
Als Gesamtergebniss dieser ersten Phase, sind die Gamification Elemente zu verstehen die sowohl den Anforderungen entsprechen und sich auch in Confluence Umsetzen lassen.
\\\\
Im zweiten Teil dieser Arbeit wird die technische Umsetzung beschrieben. Es wird dokumentiert, wie die Confluence Wiki konfiguriert wird und wie die aus Phase Eins erarbeiteten Gamification Elemente in die Wiki zu integrieren sind. Des weiteren sollen in dieser Phase Kennzahlen und Messwerte festgehalten werden. Diese Zahlen sollen genutzt werden um die Veränderung der Dokumentationsbereitschaft zu messen. Hierfür soll ein einfacher Vergleich von Alter-Zustand zu Neuer-Zustand gezogen werden. Dabei steht der Alter-Zustand für Confluence ohne Gamification und der Neuer-Zustand für Confluence mit Gamification.
\\\\
Im letzten Teil dieser Arbeit sollen die Ergebnisse aus dem Vergleich von Alter-Zustand zu Neuer-Zustand präsentiert und untersucht werden. Diese Bewertung soll Aufschluss darüber geben wie rentable es ist, Gamification Elemente für das konkrete Beispiel des Dokumentationsprozesses zu nutzen. Zum Schluss soll die Arbeit mit einem Fazit und einem Ausblick einblicke in zukünftige Projekte geben.

\section{Knowledge Management}
In diesem Kapitel sollen die Grundlagen aus dem Bereich Knowledge Management erläutert werden, die für diese Arbeit relevant sind. Das Thema Knowledge Management ist zu umfangreich um in seiner Gesamtheit für diese Arbeit in Betracht genommen zu werden. Deswegen soll sich in dieser Arbeit nur auf das Dokumentieren (Knowledge Capturing) als Untergruppe des Knowledge Management beschränkt werden. Hierzu sollen zunächst die in diesem Kapitel verwendeten Begrifflichkeiten definiert und erklärt werden. Im Anschluß werden die verschiedenen Punkte beschrieben was der generelle nutzen einer Dokumentation ist und was die Vorteile und Schwierigkeiten des Dokumentationsprozesses sind. Desweiteren sollen die Qualitätsmerkmale einer Dokumentation aufgeführt und anhand von Beispielen erklärt werden. Darauf folgen Gründe, warum Dokumentationen ungerne geschrieben werden.
\\\\
Um den Wissensaustausch durch Dokumentationen zu unterstützen muss zuerst der Dokumentationsprozess analysiert werden. Hierfür werden Recherchen betrieben die sich mit Knowledge Sharing, Dokumentation, internal Qualität oder ähnlichen Themen beschäftigen. Dadurch sollen Faktoren bestimmt werden, die sich dabei auf den Prozess auswirken. Diese Faktoren werden am Ende als Liste Zusammengefasst und dienen als Überleitung an den Gamification Teil. In den Folgenden Kapiteln wird dann ermittelt, welche Gamification Elemente sich nutzen lassen um die Faktoren positiv zu beeinflussen. Durch diese Beeinflussungen sollte sich dann auch der Dokumentationsprozess verbessern.
\\\\
Durch das untersuchen des Knowledge Capturing soll der Bereich aus verschiedenen Perspektiven betrachtet werden. Dadurch sollen nicht nur technische Schwierigkeiten des Dokumentationsprozesses aufgedeckt werden sondern vor allem auch die psychologischen Hintergründe, die das Dokumentieren so komplex machen.
\subsection{Definitionen und Begrifflichkeiten}
Hier sind die Definitionen und Begriffe erläutert wie wir sie für diese Arbeit verwenden.
\\\\
Ein \textit{Autor} ist der Verfasser einer Dokumentation. Der Autor wird meistens der Entwickler sein, da wir uns aber nicht nur auf Software-Dokumentation beschränken, können auch andere Personen wie Technische Schreiber oder Manager als Autoren in Betracht kommen. Ein Autor ist gleichzeitig immer ein \textit{Anwender} der Confluence Wiki und kann in diesem Zusammenhang deswegen auch als solcher genannt werden.
\\\\
Ein \textit{Leser} ist die Person (Entwickler, Stakeholder etc) welche die Dokumentation liest. Für den Leser wird die Dokumentation geschrieben. Er ist auch derjenige der versucht Information aus der Dokumentation zu gewinnen. Mehrere Leser werden in dieser Arbeit zusätzlich als \textit{Publikum} verstanden.
\\\\
Eine \textit{Dokumentation} oder \textit{Wissensbasis} (Knowledge base) eines Projektes bezieht sich auf alle menschliche- und maschinen Lesbare Medien die Teile von Informationen des Projektes enthalten. In dieser Arbeit werden beide Definitionen gleichgesetzt da sich beide Begriffe auf das vorhandensein von Informationen an einem Zentralen Ort beziehen (in der Confluence Wiki). Es werden keine Unterscheidungen zwischen den verschiedenen Arten von Dokumentationen getroffen wie Software-Dokumentation oder Architektur-Dokumentation. Stattdessen wird der Begriff “Dokumentation” vereinfacht und verallgemeinert genutzt, da hier die Tätigkeiten, Informationen aufnehmen in den Vordergrund zu rücken.
\\\\
Als \textit{Dokumentationsprozess} (schreiben von Dokumentation, dokumentieren) wird in dieser Arbeit jener Prozess bezeichnet, der für das Sichern von Informationen verantwortlich ist. Dabei wird nicht auf die Art der Information eingegangen um den Begriff universaler in dieser Arbeit verwenden zu können. Desweiteren wird in dieser Arbeit ebenfalls nicht festgelegt, an welchem Ort die Information gespeichert wird. Da in dieser Arbeit die Confluence Wiki als referenzpunkt genutzt wird, wird in dieser Arbeit aber davon ausgegangen dass Informationen in diese Wiki eingetragen werden.
\\\\
Der Begriff \textit{technische Schulden} wurde 1992 von Ward Cunningham vorgestellt. Technische Schulden beschreiben dabei den Vorgang, in einem Projekt diejenigen Aktivitäten zu vernachlässigen welche keine neuen Funktionalität in das Projekt bringen. 
\\\\
Zu diesen reduzierten Aktivitäten zählen meistens die Dokumentation des Projektes und das Testen von Funktionen innerhalb des Projektes. Ein Entwickler der an diesem Projekt arbeitet verschuldet sich also, indem Dokumentation und das Testen übergangsweise vernachlässigt werden. Dadurch wird kurzzeitig die Entwicklung des Projektes vorangetrieben. Diese technischen Schulden müssen aber, wie alle Schulden, zu einem späteren Zeitpunkt zurückgezahlt werden. 
\\\\
Die Rückzahlung bei technischen Schulden erfolgt durch das erneute investieren von Zeit, um das Wissen zu erarbeiten, welches während dem Projekt zwar gewonnen wurde, aber nicht rechtzeitig dokumentiert und somit verloren ging. Dies erfordert zusätzliche Anstrengungen die als \textit{technische Zinsen} zu verstehen sind. Die Rückzahlung ist somit teurer als die eigentlichen Kosten für die Dokumentation da die Rückzahlung mit den Zinsen addiert wird.

\subsection{Zweck einer Wissensbasis}
Der Zweck einer Dokumentation besteht darin, dass relevante Information zu einem Thema gesammelt, gespeichert und aufbereitet werden. Diese Informationen werden aus verschiedenen Gründen erfasst und gespeichert.

\subsubsection{Spätere Verwendung der Information für Autor}
Eine Studie von Parnin im Jahr 2010 \cite{Parnin2010} über kognitive Prozesse von Entwicklern beschreibt das Problem im Bereich der Softwareentwicklung: Ein Entwickler dessen Aufgabe darin besteht Quellcode zu generieren verliert schon gleich nach der Implementierung des Codes Informationen die für Änderungen oder Erweiterungen des Codes wichtig wären. Dieses Wissen geht verloren, solange es nicht ausreichend dokumentiert wird. Nach nur wenigen Minuten Ablenkung verliert ein Entwickler bereits seine gut durchdachten Gedankengänge \cite{Graham2004}. Nach wenigen Tagen verblasen Informationen zu ausgewählten Namen und Identifizieren. Nach Wochen verliert ein Entwickler seine Erinnerung an die Mentale Repräsentation die benötigt wird um Aufgaben in bezug auf den Quellcode zu erfüllen. Es zeigt sich also, dass die menschliche Erinnerung nicht effektiv genug Informationen und Wissen abspeichern kann.

\subsubsection{Mitarbeiter schneller in das Projekt integrieren} 
Der Prozess des Onboardings (einen neuen Mitarbeiter in das Projekt einlernen) ist ein wichtiger Prozess in der IT Branche und besonders bei Beratungsfirmen. Die Fachkenntnisse die in Form von Beratern von Kunden eingekauft werden, bestimmen den Wert der Dienstleistung gegenüber dem Kunden. Mitarbeiter wechseln häufiger nach Bedarf oder Aufgrund von spezifischen Fachkenntnissen die Projekte. Dadurch entstehen zwei Probleme für das Projekt. Verlässt ein Mitarbeiter ein Projekt so verliert es effektiv an Wissen in Form des Mitarbeiters. Wird dieses wertvolle Wissen nicht rechtzeitig gesichert, entstehen finanzielle Kosten. Die Kosten beziehen sich dabei auf erneute Zeitinvestitionen die gemacht werden müssen, um das verlorene Wissens wieder aufzuarbeiten.
\\\\
Zudem sind neue Mitarbeiter mit dem Projekt und dessen Wissenbasis (Dokumentation) nicht vertraut. Das einarbeiten eines Mitarbeiters kostet Zeit und unter umständen Personal. Um das Einlernen effizienter zu gestalten, ist eine verständliche Wissenbasis unerlässlich. Sie bietet einen Startpunkt in das Projekt für den Mitarbeiter, gibt ihm grundlegende Informationen und ermöglicht ihm Fragen im Vorfeld eigenständig zu beantworten. Desto verständlicher und strukturierter die Dokumentation, desto effizienter ist der Onboarding-Prozess.   
\\\\
Hinzu kommt das neue Mitarbeiter ihre Probleme effizienter kommunizieren können, da die Kommunikation über die Dokumentation geführt werden kann. Somit können Senior Mitarbeiter (Mitarbeiter die mit den Projekt gut vertraut sind) einfach auf die Dokumentation verweisen wenn es Unklarheiten gibt. 

\subsubsection{Erweiterbarkeit des Projektes wird gesichert}
Durch eine strukturierte Wissensbasis in form einer Dokumentation können leichter Änderungen an den Projekten durchgeführt werden. Die Informationen die hierfür aus der Dokumentation genutzt werden verhindern, das Änderungen an der falschen Stelle gemacht werden. Die Informationen helfen ebenfalls zu ermitteln, welche Lösungsansätze bereits versucht worden sind und wie erfolgreich diese waren. Es fällt ebenfalls leichter zu erkennen ob die verlangten Änderungen umsetzbar sind oder nicht.

\subsubsection{Wartbarkeit der Software wird sicher gestellt}
Das Warten und Instandhalten von Software gehört immer mehr zu den Tätigkeiten eines Entwickler. Deswegen ist es auch hier wichtig, eine verständliche Wissensbasis mit entsprechenden Informationen zu haben die erklären, wie die Software oder das System arbeitet. Dadurch können Fehlerquellen eingegrenzt bzw. identifiziert werden. Fehler die häufiger auftreten, können ebenso in die Dokumentation aufgenommen werden wie grundsätzliche Einstellungen und Konfigurationen.
 
\subsubsection{Nutzen für verteilte Projekte}
Die Dokumentation dient bei verteilten Projekten ebenfalls als Kommunikationsmedium. Dies ist ein wichtiger Punkt für die esentri AG da nicht immer alle Mitglieder des Projektes Vor Ort sind und sich dadurch längere und vor allem kompliziertere Kommunikationswege auftun. Durch das Verweisen auf die Dokumentation, können umständlicher Kommunikationskanäle wie Telefongespräche oder E-mails reduziert werden oder komplett wegbleiben. Obwohl eine Dokumentation es nicht vermag komplett auf Meetings oder Besprechungen zu verzichten, helfen sie doch durch ein gemeinsames Wissensvokabular diese Meetings effizienter zu gestalten. 

\subsection{Die Hürden des Dokumentationsprozesses}
Das schreiben von Dokumentationen ist natürlich ein Aufwand so wie das implementieren von Code auch. Dieser Prozess kostet Zeit und ist aufgrund seiner Komplexität entsprechend schwer greifbar. Die Komplexität beim Dokumentationsprozess rührt daher, das verschiedene Einflüsse in eine Dokumentation miteinfließen.
\\\\
Zunächst müssen diejenigen Informationen identifiziert werden, welche in die Dokumentation aufgenommen werden sollen. Hierfür muss der Autor entscheiden, welche Informationen wichtig oder unwichtig sind. Diese Fähigkeit zu erlangen erfordert Übung und Erfahrung. Das Kategorisieren (Wichtig/Unwichtig) von Informationen ist eine komplexe Angelegenheit, da hier sowohl der Wissensstand des Autors als auch der Wissensstand des Lesers eine entscheidende Rolle spielt. Was für den einen Trivial und ersichtlich erscheint muss für den Anderen nicht unbedingt ebenfalls so sein. Ist die Wissenslücke zwischen Autor und Leser zu groß, wirkt sich das negativ auf die Qualität der Information und somit automatisch auf die Qualität der Dokumentation aus. Die Identifizierung und die Kategorisierung von Information kann bis heute noch nicht automatisiert werden da es bis jetzt noch kein System gibt das das menschliche Urteilsvermögen ersetzen kann.
\\\\
Der nächste Faktor in der Komplexität des Dokumentationsprozesses ist die Darstellung der Information. Die Information sollte leicht verständlich, kurz und knapp dargestellt sein. (KISS-Prinzip\footnotemark
). Nicht jeder Autor ist jedoch in der Lage seine einzigartigen Gedanken so zu formulieren, das Sie als verständlich angesehen werden. Informationen kurz oder knapp zu halten ist ebenfalls nicht so trivial wie man anfangs vermuten lässt. Schließlich müssen Informationen kompakt und auf den Punkt gebracht werden. Die Essenze einer Information zu sichern muss ebenfalls dem menschlichen Urteilsvermögen überlassen werden. Unterstützung durch Tools oder Systeme gibt es auch hier nicht.
\footnotetext{\url{http://www.community-of-knowledge.de/beitrag/keep-it-simple-stupid-leichtgewichtiges-wissensmanagement-mit-wikis/}}
\\\\
Die identifizierten und ausformulierten Informationen müssen anschließend in eine Form und Struktur übernommen werden, die logisch und verständlich für das entsprechende Publikum ist. 

\begin{description}
   \item[Dabei gilt es die Dokumentation so zu designen:]~\par
   \begin{itemize}
      \item Informationen sollten leicht und schnell zu erkennen sein.
      \item Informationen sollten säuberlich voneinander getrennt sein. 
      \item Informationen sollten entsprechend geordnet sein.
   \end{itemize}
\end{description}
Die Komplexität einer Dokumentation beschränkt sich aber nicht nur auf die entsprechenden Informationen. Jede Dokumentation muss je nach Publikum und Themengebiet unterschiedlich gestaltet werden. Zwar gibt es Standards und Normen, wie z.B der arc42-Standard als Dokumentationsstandard für Architekturen. Dennoch können die vielseitigen Sichten nicht alle in einen Standard vereint werden.
\\\\
Hinzu kommt dass das Schreiben von Dokumentationen ein tiefes Verständnis des Themengebietes voraussetzt über das dokumentiert wird. Im Falle der Softwareentwicklung, die hier als Beispiel verwendet werden soll, bedeutet es das der Implementierte Quellcode verstanden werden muss. Im Idealfall dokumentiert der entsprechende Entwickler selbst den Code, ist dies jedoch nicht der Fall und die Dokumentation wird von einem Dritten geschrieben, gehen schonmal Hintergrundinformationen die eventuell für das Projekt relevant sind verloren. Somit ist es also immer besser wenn der entsprechende Entwickler einer Funktion oder Komponente (oder ein andere teil des Quellcodes) dessen Dokumentation übernimmt. Dies ist aber nicht immer möglich, da die Zeit fehlt und erfahrene Entwickler mehr damit beschäftigt sind, neue Funktionalität zu implementieren und somit das dokumentieren unerfahrenen Entwicklern überlassen wird. 
\\\\ 
Ein weiteres schwerwiegendes Problem ist auch, dass die Dokumentation einer der ersten Aktivitäten ist die bei hohen Zeitdruck vernachlässigt wird. Bei Projekten mit einem straffen Zeitplan und näherrückenen Deadlines werden so technische Schulden gemacht um so den vorgegebenen Zeitplan einhalten zu können. Diese Schulden zu erfassen ist schwierig da sie stark kontextabhängig sind. Außerdem müssen ebenfalls die technischen Zinsen erfasst und berücksichtigt werden. Auch diese Angaben sind schwer zu erfassen und somit wird verhindert dass die gesamten Schulden sachgemäss zurückgezahlt werden können.

\subsection{Vorteile des Wissensaustausch}
Dennoch zeigen Studien das eine Dokumentation trotz Zeitaufwand und Anstrengungen sich für Firmen rentiert. In einem Experiment von Eirik Tryggeseth zu den Auswirkungen von Dokumentationen auf die Wartung von Software zeigt sich z.B, dass Änderungen an Software Projekten durch Dokumentationen sich effizienter umsetzen lassen. die Lösungsansätze und Änderungen waren dank Hintergrundinformationen zusätzlich qualitativ besser \cite{Tryggeseth1997}.
\\\\
Bei diesem Experiment wurden zwei Gruppen von Studenten mit Änderungen an einem Software-Projekt beauftragt. Eine Gruppe verwendete dabei eine Dokumentation, während die zweite Gruppe lediglich den Quellcode des Projektes zur Verfügung hatte. Die Gruppe ohne Dokumentation benötigte 21,5\% mehr Zeit ihre Änderungen umzusetzen als ihre Kollegen die über eine Dokumentation verfügten \cite{Tryggeseth1997}.
\\\\
Das hier erwähnte Experiment zeigt deutlich den größten und wichtigsten Nutzen einer vernünftigen Wissensbasis, das generieren von Zeiterspanissen. Dabei wird sich aber nicht auf die Entwicklungszeit bezogen. Die Entwicklungszeit eines Projektes soll sich in diesem Zusammenhang auf die gesamte Zeitspanne beziehen, welche das planen, implementieren, dokumentieren, testen und inbetriebnehmen eines entwickelten Projektes beinhaltet. Oder bei Agilen Projekten auf die Zeit aller durchgeführten Sprints. Diese Entwicklungszeit wird natürlich nicht beschleunigt. Die Zeitersparnis macht sich erst dann bemerkbar, Wenn das hier gefundene Wissen in anderen Projekten zur einer Verkürzung deren Entwicklungszeit führt. Dadurch wird das gesammelte Wissen also wiederverwendet und führt dadurch zu einer höheren Qualität von Projekten, bei denen das Wissen verwendet wird.
\\\\
Als solches kann eine Dokumentation als Investition in die Zukunft gesehen werden. Die Erfahrungen und das Wissen das bei vergangenen Projekten gewonnen wurde, führt zur besseren Qualität und von einer Senkung der Entwicklungszeit bei zukünftigen Projekten.
\\\\
Ein weiterer Vorteil einer stämmigen Wissensbasis ist das Vermarkten des generierten Wissens. Wie bereits erwähnt ist die esentri AG eine Beratungsfirma. Das Kapitel einer Beratungsfirma liegt in den Beratern/innen und deren Fachkenntnisse. Wird das Wissensmanagement ausgebaut und den Mitarbeitern leichter Zugang und ein schnellerer Austausch von Informationen ermöglicht, steigert sich auf lange Sicht gesehen nicht nur die Qualität der Beratungen sondern hebt esentri gegenüber seinen Konkurrenten ab.

\subsection{Qualitätsmerkmale einer Dokumentation}
In diesem Kapitel soll geklärt werden was die Merkmaler einer qualitativ hochwertigen Dokumentation sind. Diese Merkmale zu verstehen ermöglicht es uns, die konkreten Faktoren zur Beeinflussung des Dokumentationsprozesses zu ermitteln.
\subsubsection{Vollständigkeit}
Die Vollständigkeit sagt aus, ob auch wirklich alle Informationen erfasst wurden die für die Dokumentation relevant sind. Hierbei gilt es für den Autor sich ständig zu entscheiden, welche Informationen sind wichtig, welche nicht. Ist die Information ausreichend oder werden noch weitere Informationen zu einem Punkt benötigt. Letzten endes ist eine Dokumentation die Summe seiner Information \cite{Prause2013}.
\\\\
Fehlen Informationen muss der Leser erst darauf kommen das nicht alle Informationen vollständig sind. Das kann er unter Umständen nicht beurteilen. Durch das suchen der entsprechenden Informationen geht wertvolle Zeit verloren.

\subsubsection{Veränderbarkeit}
Die Veränderbarkeit sagt aus, wie einfach Informationen ausgetauscht oder modifiziert werden können. Änderungen innerhalb eines Projektes oder eines Produktes müssen natürlich auch in die entsprechende Dokumentation eingetragen werden. Dabei bezieht sich die Veränderbarkeit nicht nur auf technischer Ebene durch Wikis (wie leicht etwas in der Wiki zu ändern ist). Es geht auch darum, an wie vielen Stellen Änderungen in der Dokumentation durchgeführt werden müssen. Desto zentraler und redundanter die Informationen vorliegen, desto leichter und sichere lässt sich die Dokumentation ändern \cite{Prause2013}.

\subsubsection{Aktualität}
Ist die Dokumentation noch auf den Aktuellsten Stand? Oder wurden Veränderungen durchgeführt die noch nicht in der Dokumentation erfasst wurden. Es ist schwer alle Änderungen im überblick zu behalten und dafür zu sorgen, das alle Informationen so Aktuell wie möglich sind. Hier gibt es einen Zusammenhang zwischen Aktualität und Veränderbarkeit. Desto Veränderbarer die Dokumentation (liegen Informationen zentral vor), desto leichter ist es die Dokumentation Aktuell zu halten \cite{Prause2013}.

\subsubsection{Eindeutigkeit}
Mit diesem Merkmal soll beschrieben werden, wie eindeutig die Informationen innerhalb der Dokumentation sind. Es geht darum zweideutigkeit zu vermeiden und Missverständnisse zu eliminieren. Eindeutigkeit hilft dem Leser dabei, schnell zu bestimmen ob die Information für ihn relevant ist und hilft bei der Wissensübertragung \cite{Prause2013}.

\subsubsection{Identifizierbarkeit}
Wie leicht können Informationen identifiziert werden und wie macht man sie unterscheidbar. Gleiche Informationen sollten immer in gleicher Form in die Dokumentation aufgenommen werden. Dadurch lassen sich Informationen anhand Ihrer Form identifizieren anstatt anhand des Inhaltes. Dadurch wird es Lesern erleichtert, die unnötigen Informationen zu überfliegen und macht es dadurch effizienter die Information zu finden die vom Leser benötigt werden \cite{Prause2013}.

\subsubsection{Verständlichkeit}
Dieser Punkt ist sehr wichtig. Die Hauptaufgabe einer Dokumentation ist das Verteilen von Wissen. Eine Dokumentation muss an dieser Eigenschaft gemessen werden. Die Informationen müssen klar abgeschlossen sein. Sie sollten nicht zu viel Information enthalten da die Aufnahmefähigkeit der Lesser begrenzt ist. Die Informationen sollten nach dem KISS-Prinzip (“Keep it short and simple”) gestaltet werden \cite{Prause2013}.

\subsubsection{Konsistenz}
Die Konsistenz beschreibt das Fehlen von Widersprüchen. Eine Dokumentation ist nur dann verständlich wenn es keine Informationen gibt die einander widersprechen. Generell sollte eine Dokumentation immer etwas neues hinzufügen. Gibt es mehrere Informationen die sich widersprechen müssen diese entfernt werden \cite{Prause2013}.

\subsection{Gründe für die Vernachlässigung des Dokumentationsprozesses}
Prause und Dudrik haben 2012 eine Umfrage von Expertenmeinungen durchgeführt. Dabei sollte ermittelt werden, ob die Experten Schwierigkeiten bzw. Probleme bei der Dokumentation und dem Architektur Design in Agiler Software Entwicklung sahen. Desweiteren sollte analysiert werden, was der Ursprung dieser Probleme ist. Dies sollte die Grundlage schaffen um Lösungsansätze für diese Probleme zu entwickeln. Insgesamt wurden in dieser Befragung 37 Experten interviewt \cite{Prause2012}.
\\\\
Die durchgeführte Befragung ergab folgende Ergebnisse wie in Tabelle \ref{ProblemTabelle} zu sehen ist.
\begin{table}[htb]
\begin{tabular}{|l|c|}\hline
\rule{0pt}{15pt}  & \textbf{Experten in Prozent}
\\
\hline
\rule{0pt}{15pt} \textbf{Alle Experten} & 74\%\\ 
\hline
\rule{0pt}{15pt} \textbf{Experten aus Industrie} & 100\% \\
\hline
\rule{0pt}{15pt} \textbf{Experten aus Akademia} & 68\%\\
\hline
\rule{0pt}{15pt} \textbf{Experten des frühen SDLC} & 70\%\\
\hline
\rule{0pt}{15pt} \textbf{Experten des späten SDLC} & 76\%\\
\hline
\rule{0pt}{15pt} \textbf{Experten aus der Entwicklung} & 74\%\\
\hline
\rule{0pt}{15pt} \textbf{Experten aus dem Management} & 78\%\\
\hline
\rule{0pt}{15pt} \textbf{Experten aus der Theorie} & 66\%\\
\hline
\rule{0pt}{15pt} \textbf{Experten mit weniger als 5 Jahren Erfahrung} & 88\%\\
\hline
\rule{0pt}{15pt} \textbf{Experten mit 5 - 10 Jahren Erfahrung} & 82\%\\
\hline
\rule{0pt}{15pt} \textbf{Experten mit 10 - 20 Jahren Erfahrung} & 71\%\\
\hline
\rule{0pt}{15pt} \textbf{Experten mit über 20 Jahren Erfahrung} & 40\%\\
\hline
\end{tabular}
\caption{Prozentsatz der Experten, die ein Problem im Dokumentationsverhalten sehen.}
\label{ProblemTabelle}
\end{table}
Die Tabelle \ref{ProblemTabelle} ist ein Auszug aus der Befragungen der Experten. Die Experten wurden dabei in verschiedene Gruppen aufgeteilt. Die Aufteilung erfolgte dabei nach ihren Tätigkeitsumfeld, Industrie oder Akademisch, nach ihren persönlichen Interesse im Entwicklungslebenszyklus (SDLC steht für Software Development Life Cycle) und nach ihrer Berufserfahrung.
\\\\
Wie in Tabelle \ref{ProblemTabelle} zu sehen ist, gaben viele der Experten zu, Probleme bei der Dokumentation zu sehen. Unabhängig von ihrem Bereich stimmten mehr als 50\% aller Befragten Experten zu, dass der Dokumentationprozess in Agilen Projekten Probleme bereiten würden. Mit einer einzigen Ausnahme. Nur knapp 40\% der Experten mit mehr als 20 Jahren Berufserfahrung sahen Probleme beim Thema Dokumentation. Eine mögliche Begründung dafür könnte sein, das Entwickler mit mehr als 20 Jahren Berufserfahrung andere Tätigkeiten in Projekten zugeteilt werden und sie deswegen weniger Probleme sehen als ihre jüngeren Kollegen \cite{Prause2012}.
\\\\
Eine weitere Erkenntnis die in dieser Befragung deutlich wurde ist die diskrepanz zwischen Industrie und Akademia. Während alle Experten aus der Industrie (100\%) die Forschungsfrage mit ja beantworteten gaben nur 68\% der Experten aus dem Umfeld der Akademia ihre Zustimmung. Dies mag wohl daran liegen, das Projekte in Industrie bzw. Akademia anderen Standards für Erfolg oder Misserfolg unterliegen.
\\\\
Die Befragung der Experten zeigt, dass die Mängel bei Dokumentationen durchaus nichts ungewöhnliches sind. Sie zeigen ebenfalls deutlich auf, wie komplex und ungreifbar das Problem von mangelnden Dokumentationen ist. Obwohl diese Befragung auf Grundlage von Agilen Prozessen basiert, glauben rund 88\% der Befragten nicht das diese Probleme ausschließlich in Agilen Projekten zu finden sind. Stattdessen sind diese Probleme sowohl bei nicht-Agilen als Agilen Projekten nachweisbar \cite{Prause2012}. Was die Wichtigkeit dieser Arbeit nochmals verdeutlicht.
\\\\
Im nächsten Teil wird betrachtet, welche Ursachen die Experten für die Probleme beim Dokumentationsprozess vermuten.
\subsection{Ursachen für mangelnde Dokumentation}
Als die Experten gefragt wurden, welche möglichen Ursachen dieses Verhalten (Wenig bzw. schlechte Dokumentation) zugrunde liegt, wurden dafür von den Experten verschiedene Gründe genannt. Diese sind in Tabelle \ref{UrsachenTabelle} zusammengefasst. Die Problemursachen die hier erwähnt werden, sind nicht ausschließend Exklusiv. Das bedeutet dass die Experten auch mehrere der Unten aufgeführten Ursachen genannt haben können.
\begin{table}[htb]
\begin{tabular}{|l|c|}\hline
\rule{0pt}{15pt}\textbf{Mögliche Ursachen für mangelnde Dokumentation}  & \textbf{Experten in Prozent}
\\
\hline
\rule{0pt}{15pt} Einige Entwickler legen zuwenig Wert auf Qualität & 49\% \\ 
\hline
\rule{0pt}{15pt} Einige Entwickler wissen nicht wie sie dokumentieren sollen & 46\%\\
\hline
\rule{0pt}{15pt} Einige Entwickler haben nicht die Zeit für Dokumentationen & 46\%\\
\hline
\rule{0pt}{15pt} Dokumentation und Entwurf wird zuwenig beachtet & 46\%\\
\hline
\rule{0pt}{15pt} Qualitätsziele werden nicht gesetzt & 35\%\\
\hline
\rule{0pt}{15pt} Einzelner Entwickler hat geringen persönlichen Nutzen & 41\%\\
\hline
\rule{0pt}{15pt} Andere Projektziele sind wichtiger & 19\%\\
\hline
\end{tabular}
\caption{Experten geben mögliche Ursachen für mangelnde Dokumentation an.}
\label{UrsachenTabelle}
\end{table}
Die Experten waren der Ansicht (46\%), das die Entwickler nicht wissen wie sie richtig zu dokumentieren haben. Einige Experten (49\%) waren der Ansicht, das Entwickler nicht genügend wert auf Qualität setzen würden. 46\% Meinen es Liege daran das die Entwickler entweder keine Zeit hätten sich um Dokumentation zu kümmern oder sie wird nicht explizit berücksichtigt. Die Experten vermuten auch (35\%) das Qualitätsziele nicht deutlich genug vorgegeben werden und der Entwickler deswegen seine Aufgaben nicht erfüllen kann. Andere Experten glauben, dass zu wenig Vorteile für den einzelnen Entwickler entstehen (41\%) wenn dieser Dokumentiert und das andere Projektziele (erreichen einer Deadline) wichtiger sind (19\%) \cite{Prause2012}.
\\\\
Diese sieben Ursachen wurden von den Experten als Begründung für mangelnde oder nicht enthaltene Dokumentation genannt. Wie wir aber bereits am Anfang dieses Kapitel erwähnt hatten, steht für uns das Knowledge Capturing im Vordergrund. Aus diesem Grund sind diese Zahlen und Einschätzungen für diese Arbeit nicht exakt zutreffend. Dennoch geben sie Interessante Einsichten und Ansätze.    
\begin{description}
   \item[So müssen die drei folgenden Ursachen genauer betrachtet werden:]~\par
   \begin{itemize}
      \item Einige Entwickler haben nicht die Zeit für Dokumentationen.
      \item Einzelner Entwickler hat geringen persönlichen Nutzen.
      \item Einige Entwickler wissen nicht wie sie dokumentieren sollen.
   \end{itemize}
\end{description}
Es soll nach den Gründen für diese Ursachen gesucht werden. Durch das genauere Untersuchen dieser Punkte können Ansätze gefunden werden, die beeinflusst werden müssen um das Verhalten und den Dokumentationsprozess zu optimieren. 

\subsection{Faktoren zum beeinflussen des Dokumentationsprozesses}
Wie sich aus den vorangegangenen Kapitel gezeigt hat, besteht für die Mitarbeiter ein zu geringes Interesse sich für den Dokumentationsprozess einzusetzen. Im Allgemeinen tut sich der Mensch schwer, eine Tätigkeit auszuführen die ihm sinnlos erscheint. Und als solche wird der Dokumentationsprozess von vielen Mitarbeitern wahrgenommen. Dies ist nachvollziehbar, wenn man bedenkt dass der größte Vorteil von Dokumentationen, sich auf zukünftige Vorhaben und deren Entwicklung beziehen. Es gibt also zu wenig persönlichen Nutzen für einen Mitarbeiter sich am Dokumentationsprozess zu beteiligen. Dies kann anhand des Verhaltensmodell\footnotemark von BJ Fogg erklärt werden.
\footnotetext{\url{https://www.youtube.com/watch?v=AdKUJxjn-R8}}
\\\\
Das Verhaltensmodell von BJ Fogg (Abb.\ref{VerhaltensmodellBild}) beschreibt die Ursachen für menschliches Verhalten in einer einfachen Formel. Das entsprechende Verhalten ergibt sich aus den drei Teilen, \enquote{motivation}, \enquote{skill} (Fähigkeit) und \enquote{trigger} (Auslöser). Mit diesen drei Punkten lässt sich ermitteln, ob ein Verhalten auftritt oder nicht. Das Modell kann über die einfache Formel \( Verhalten = Motivation + Fähigkeit + Auslöser \) dargestellt werden.
\subsubsection{Verhaltensmodell im Überblick}
Sehen wir uns das Verhaltensmodell, das in Abb.\ref{VerhaltensmodellBild} dargestellt ist, anhand eines einfachen Beispiels an. Das Verhalten \enquote{Bringe Müll raus} würde sich durch das Verhaltensmodell wie folgt darstellen. 
\begin{description}
   \item[Das Verhalten tritt auf:]~\par
   \begin{itemize}
      \item Wenn genügend Lust vorhanden ist, den Müll rauszubringen(Motivation).
      \item Wenn man in der Lage ist, den Müll rauszubringen(Fähigkeit).
      \item Und wenn man merkt das der Müll voll ist(Auslöser).
   \end{itemize}
\end{description}  
Sind alle drei Merkmale (motivation, skill und trigger) vorhanden, tritt das entsprechende Verhalten (Bringe Müll raus) auf. Dabei ist es nicht unbedingt erforderlich, dass das Motivation- und Fähigkeiten-Merkmal immer hoch sein müssen. Es kommt lediglich darauf an ob die \textit{Verhaltensschwelle} überschritten wird oder nicht (Abb.\ref{VerhaltensmodellBild}). Tätigkeiten die sowohl schwierig umzusetzen sind und bei denen jegliche Motivation fehlt führen dazu, dass die Auslöser für das Verhalten fehlschlagen (Triggers fail here, Abb.\ref{VerhaltensmodellBild}), da hier die Verhaltensschwelle nicht überschritten wird. Dadurch wird das Verhalten nicht hervorgebracht. Das gleiche gilt auch für den umgekehrten Fall. Wird die Verhaltensschwelle überschritten, so wird das Verhalten verursacht. 

\subsection{Die Bestandteile von Verhalten, Motivation, Fähigkeit und Auslöser}
Betrachten wir nun den Dokumentationsprozess als unser vorgegebenes Verhalten (Informationen aufnehmen und speichern), und untersuchen wie dieser Prozess sich auf die drei Merkmale Motivation, Fähigkeit und Auslöser abbilden lässt. Über diese Vorgehensweise wird weitere Einsicht in die menschliche Komponente des Dokumentationsprozesses gewonnen.

\subsubsection{Die Motivation eines Verhaltens}
Die Motivation beim Dokumentationsprozess ist wie wir bereits festgestellt haben sehr gering. Einer der wichtigsten Gründe hierfür liegt wohl daran, dass sich der Nutzen dieses Prozesses erst in der Zukunft bemerkbar macht. Dieser Nutzen zeigt sich anhand eines geringeren Zeitinvestments der Mitarbeiter in zukünftigen Projekten. Das geringere Zeitinvestment wird als solches aber nur gering vom Mitarbeiter wahrgenommen und bietet dem Autor der Dokumentation einen zu geringen persönlichen Nutzen. Menschen bevorzugen aber sofortige und schnelle Gratifikation die sich über diesen Nutzen (geringes Zeitinvestment) nicht erfüllen lässt \cite{Kuhl2009}.
\\\\
Dies erklärt, warum andere Beschäftigungen dem Dokumentationsprozess bevorzugt werden. Da andere Beschäftigungen eine Gratifikation leichter ermöglichen als der Dokumentationsprozess. Da der Nutzen von Zeitersparnis zu unscheinbar und unbefriedigend ist, müssen an dieser Stelle Alternativen eingesetzt werden. Hierfür muss die menschliche Motivation eingehender betrachtet werden.
\\\\
Die Natur der Motivation ist ein umstrittenes Thema in der Forschung und Wissenschaft \cite{Eyal2014}. Um dieses Feld für diese Arbeit Einzugrenzen wird jedoch die Argumentation von BJ Fogg verwendet, da er als einer der führenden Experten in der Verhaltens- und Gewohnheitsforschung gilt. Da sich die Verhaltensforschung mit unserem eigentlichen Ziel überschneidet (Dokumentationprozess als Verhalten zu verbessern), ist seine Argumentation von großer Bedeutung für diese Arbeit. BJ Fogg argumentiert dass es drei Kernpunkte gibt, welche die Motivation in Bezug auf entsprechendes Verhalten ausmacht \cite{Eyal2014}. 
\begin{description}
   \item[Diese Kernprinzipien lauten:]~\par
   \begin{itemize}
      \item Menschen suchen nach Vergnügen und vermeiden Schmerz.
      \item Menschen suchen nach Hoffnung und vermeiden Furcht.
      \item Menschen suchen nach sozialer Akzeptanz und vermeiden Zurückweisung.
   \end{itemize}
\end{description}
Durch die Verwendung und Erweiterung dieser Kernprinzipien \cite{Eyal2014}, kann man die Wahrscheinlichkeit erhöhen, dass eine Person einer bestimmten Aktion nachgeht oder nicht. Im Falle des Dokumentationsprozesses können wir uns auf den ersten dieser drei Punkte stützen, Menschen suchen nach Vergnügen und vermeiden Schmerz. Um dieses Kernprinzip für den Dokumentationsprozess zu nutzen, soll der Begriff des Vergnügens einmal genauer untersucht werden.

\subsubsection{Vergnügen und Belohnung}
Das frühe psychologische Konzept des Vergnügens (Pleasure) und auch das Lustprinzip (pleasure principle) werden als Feedback System gesehen. Im Detail handelt es sich bei Vergnügen um:
\\\\
\enquote{einen positiven Feedback-Mechanismus, der den Organismus motiviert, in Zukunft die Situation, die er gerade als angenehm empfunden hat, wiederherzustellen und Situationen zu vermeiden, die in der Vergangenheit Schmerzen verursacht haben} \cite{Freud2015}.
\\\\
Nun muss die Frage beantwortet werden, wie wir dieses Feedback-Mechanismus verwenden können? Die Antwort lautet: Über das bereitstellen einer entsprechenden Belohnung. Denn für die meisten Menschen ist eine Belohnung etwas begehrtes, weil sie eine bewusste Erfahrung des Vergnügens erzeugt \cite{Berridge2009}. Anstelle den Nutzen einer Dokumentation als einzige Motivationsquelle zu verwenden, ist es also erforderlich, eine Alternative, in form einer zusätzlichen Belohnung, zur Verfügung zu stellen.
\\\\
Dabei ist die Art der Belohnung für jeden Menschen anders und damit \enquote{subjektiv}. Wissenschaftler sprechen an dieser Stelle von \enquote{subjektiven Belohnugswerten} \cite{Bonhoeffer2011}. Jeder Mensch misst sich in seinem eigenen Belohnungssystem das ein Teil des Gehirns ist, seine ihm persönlichen Werte zu. Durch die Gamification müsste nun ein generell gültige Belohnung entworfen werden, die für jeden bzw. die meisten Mitarbeiter der esentri AG einen hohen subjektiven Belohnungswert hat. Dadurch würde sich die Motivation steigern lassen und es würde Mitarbeiter eher anspornen sich am Dokumentationsprozess zu beteiligen oder ihn gar von sich aus auslösen.
\subsubsection{Die Fähigkeit ein Verhalten umzusetzen}
Die Fähigkeit beschreibt im Verhaltensmodell von BJ Fogg, ob es sich um eine schwierige oder einfache Aktion handelt aus der sich das entsprechende Verhalten ergibt. In den vorangegangenen Kapiteln hat sich auch hier deutlich gezeigt, das der Dokumentationsprozess mit komplexen Eigenschaften versehen ist, was bedeutet das er in der Fähigkeiten Skala des Verhaltensmodells eher Hoch und damit als schwer einzustufen ist. Wenn unser gewünschtes Verhalten hervortreten soll, ist es erforderlich den Dokumentationsprozess zu vereinfachen. Für die Fähigkeit gibt es ebenfalls Merkmale über die der Schwierigkeitsgrad beeinflusst werden kann.
\begin{description}
   \item[Die Fähigkeitsmerkmale sind:]~\par
   \begin{itemize}
      \item \textbf{Zeit}: Wie lange benötigt man für die entsprechende Aktivität?
      \item \textbf{Geld}: Wie viel Geld kostet die Aktivität?
      \item \textbf{Physical Anstrengung}: Wie anstrengend ist diese Aktivität?
      \item \textbf{Verständnis}: Wie leicht ist die Aktivität zu verstehen?
      \item \textbf{Soziale Abweichung}: Wie viele andere Personen führen die Aktivität aus?
      \item \textbf{Übung}: Wie geübt ist man, die Aktivität durchzuführen?
   \end{itemize}
\end{description}
Diese Fähigkeitsmerkmale sind ebenfalls in Abbildung \ref{FähigkeitsmerkmaleBild} zu sehen. Durch das manipulieren und beeinflussen dieser Fähigkeitsmerkmale, wird der Schwierigkeitsgrad der Aktivität (Informationen speichern) regulierbar. Durch die Reduzierung der Schwierigkeit auf ein niedrigeres Niveau, wird die Wahrscheinlichkeit den Dokumentationsprozess auszuführen gesteigert. Dabei müssen nicht alle sechs Fähigkeitsmerkmale (Zeit, Geld, Anstrengung, Verständnis, Soziale Abweichung und Übung) berücksichtigt werden. Die Merkmale Zeit, Verständnis und soziale Abweichung sollen für diese Arbeit in Betracht gezogen werden. 

\subsubsection{Fähigkeitsmerkmal Zeit}
Zunächst soll das Fähigkeitsmerkmal Zeit untersucht werden. Hier gilt es zu prüfen, ob der Dokumentationsprozess nicht beschleunigt werden kann. Um den Dokumentationsprozess zu beschleunigen, ist es zunächst erforderlich dass der Prozess analysiert wird \cite{Hauptly2008}.
\begin{description}
   \item[\parbox{\textwidth}{Um den Dokumentationsprozess sachgemäß zu analysieren müssen die Folgenden Fragen beantwortet werden: \normalfont\vspace{0.5ex}}]~\par
   \begin{enumerate}
      \item Was versucht der Anwender mit dem Dokumentationsprozess zu erreichen?
      \item Welche Schritte müssen vom Anwender unternommen werden, damit er sein Ziel erreicht?
      \item Welche dieser Schritte können entfernt werden ohne vom Ziel des Anwenders abzuweichen?
      \item Welche Schritte können modifiziert werden um sie zu verkürzen?
   \end{enumerate}
\end{description}
Wenn diese Fragen sauber und sachgerecht beantwortet werden können, kann der Dokumentationsprozess optimiert werden. Durch eine Verkürzung der Dauer des Dokumentationsprozesses werden Mitarbeiter eher dazu verleitet, diesen Prozess zu nutzen. Die Analyse des Dokumentationsprozesses wird in kommenden Kapiteln behandelt.

\subsubsection{Fähigkeitsmerkmal Verständnis}
Das Fähigkeitsmerkmal Verständnis kann ebenfalls genutzt werden um den Dokumentationsprozess zu vereinfachen. Sind die einzelnen Schritte des Prozesses erkannt und optimiert, ist es wichtig sie so zu designen, dass sie im Idealfall selbsterklärend und leicht verständlich sind. Dabei ist es notwendig, die Schritte aus denen der Dokumentationsprozess besteht auf ihre kleinst mögliche Handlung zu reduzieren.
\\\\
An dieser Stelle kann zusätzlich die Gamification genutzt werden. Gamification wird nämlich nicht ausschließlich genutzt, um Verhalten zu verändern. Gamification kann ebenfalls genutzt werden, um das erlernen neuer Skills zu fördern. Dies kann nun verwendet werden um den neuen optimierten Dokumentationsprozess für die Anwender zugänglicher zu gestalten.
\\\\
Wie die Gamification sich nun im Detail auf den überarbeiteten Dokumentationsprozess auswirken kann und welche Elemente hierfür verwendet werden sollen, wird in den kommenden Kapiteln erläutert und gezeigt.

\subsubsection{Fähigkeitsmerkmal soziale Abweichung}
Menschen sind Einzelwesen und soziale Wesen zugleich \cite{Vester2009}. Sie suchen Geborgenheit, Schutz und Anerkennung innerhalb einer sozialen Gemeinschaft. Wie wir bereits bei der Motivation gesehen haben, strebt der Mensch nach sozialer Akzeptanz und versucht, soziale Zurückweisungen zu vermeiden. Dieses Prinzip bildet ebenfalls die Grundlage für dieses Fähigkeitsmerkmal.
\\\\
Die soziale Akzeptanz die der Mensch anstrebt bringt sie dazu, ein gewisses Verhalten hervorzurufen, wenn dieses Verhalten von einer Gemeinschaft der man beitreten möchte als positiv empfunden wird. Bezogen auf den Dokumentationsprozess lässt sich also behaupten, das wenn eine Gemeinschaft (wie eine Firma), den Dokumentationsprozess als ein positives und gewünschtes Verhalten ansieht, es die Wahrscheinlichkeit erhöht dass Mitglieder dieser Gemeinschaft, also die Mitarbeiter, dieser Tätigkeit eher nachgehen \cite{Eyal2014}.
\\\\
Um dieses Fähigkeitsmerkmal nutzen zu können, ist es erforderlich den Dokumentationsprozess so zu entwickeln, das dieser einem sozialen Netzwerk ähnelt. Es geht bei diesem Ansatz darum, den Mitarbeitern zu verdeutlichen, wer in der Gemeinschaft bzw. Netzwerk dokumentiert und wie häufig diese Person dokumentiert. Die Inhalte und Informationen die in die Wiki von den Autoren eingetragen werden, sollen als “Eigentum” dieses Autors verstanden werden. Dies soll dazu führen das Autoren sich um ihre Informationen kümmern, auch wenn sie bereits in der Wiki besteht. Es wird Sorgsamkeit durch Verantwortlichkeit generiert. Autoren und Leser sollten auf dieser Ebene in der Lage sein, einen sozialen Austausch von Ideen oder Verbesserungen zu betreiben und es sollte ihnen ermöglicht werden sich gegenseitig Lob und Anerkennung zu übermitteln.
\\\\
Durch diese Änderungen am Dokumentationsprozess, kann die Anteilnahme der Mitarbeiter sich am Prozess zu beteiligen gesteigert werden. Aus der Wiki wird also mehr als nur eine normale Wiki, sie wird zu einem sogenannten Reputation Based System \cite{Prause2013}. Durch den Ruf den sich Anwender der Wiki und somit die Autoren erwerben, können diese Anwender motiviert und engagiert werden.
\\\\
Ein Beispiel für ein Reputation Based System wäre z.B \enquote{Stackoverflow}. Stackoverflow ist die möglicherweise Bekannteste Frage-Antwort Website der Welt. Ein wichtiger Punkt der zum Erfolg von Stackoverflow beiträgt ist das dort verwendete Reputation System. Die Community entscheidet über \textit{votes}, ob eine Antwort die auf eine Frage folgt hilfreich ist oder nicht. Dadurch werden je nach Anzahl der Votes für den Antwortsteller Punkte generiert. Und diese Punkte ermöglichen den Mitglieder von Stackoverflow im Level zu steigen und sich so ihren Ruf innerhalb der Community aufzubauen. Hierdurch werden die Mitglieder der Community engagiert und es wird immer die beste Antwort gefunden. Durch diesen simplen Ansatz, wird die Tätigkeit eine Frage zu beantworten zur lohnenden Erfahrung.

\subsection{Der Auslöser eines Verhaltens}
Der Auslöser beschreibt im Verhaltensmodell das Auftreten einer bestimmten Situation, die dazu führt, dass das Verhalten gestartet wird, wenn genügend Motivation und ausreichend Fähigkeiten vorhanden sind (Verhaltensschwelle wird überschritten). Ist die Verhaltensschwelle in dem Moment in dem der Auslöser auftritt jedoch zu gering, wird das Verhalten nicht ausgelöst. Für den Dokumentationsprozess als unser gewünschtes Verhalten bedeutet es, das wir nach einer bestimmten Situation suchen müssen, die das Abspeichern von Information auslöst. Liegt ein betreffender Auslöser nicht vor oder kann er nicht eindeutig genug identifiziert werden, muss man alternativ einen Auslöser selbst designen.
\\\\
Bei den Auslösern ist zu beachten, dass es sie in zwei verschiedenen Formen gibt. So lassen sich \enquote{innerliche Auslöser}(internal Trigger) und \enquote{äußerliche Auslöser}(external Trigger) unterscheiden \cite{Eyal2014}.
\\\\
Die äußerlichen Auslöser sind daran zu erkennen, dass die Information die den Handlungsschritt enthält auf den Auslöser abgebildet ist. Der Auslöser wird somit zu einem Aufruf, eine Handlung oder Aktion auszuführen. Und die Handlung die aufgefordert wird befindet sich innerhalb des Aufruf. In Abb.\ref{externaltriggerBild} sind einige dieser Auslöser zusammengefasst, jedoch sind dies nicht alle existierenden Auslöser.
\\\\
Bei den äußerlichen Auslösern handelt es sich z.b um verschieden Knöpfe in entweder digitaler oder mechanischer Form. Die Darstellung der Apps auf den Screens unsere Smartphones sind ebenfalls äußerliche Auslöser. Diese Auslöser erlauben es die folgende Aktivität (Video abspielen, E-mail lesen, Nachricht senden etc.) direkt zu starten. Diese Auslöser gibt es in verschiedensten Formen und sind klar verständlich und sehr intuitiv. Mit anderen Worten sie sind kinderleicht zu erkennen und zu bedienen\cite{Eyal2014}.
\\\\
Die innerlichen Auslöser beziehen sich nun mehr auf die menschliche Psychologie. Während die äußerlichen Auslöser sich durch technische Lösungen darstellen, sind die innerlichen Auslöser die aus dem inneren eines jeden Menschen kommen. Gemeint sind damit unter anderem, Emotionen, andere Menschen, spezielle Situationen, Orte oder gar Routinen (Abb.\ref{internalTriggerBild}). 
\\\\
Die innerlichen Auslöser werden durch die mentale Haltung (Psychologie) des Menschen verursacht. Die Information, welche Handlung auf diese Art von Auslöser folgt, wird durch eine Assoziation des entsprechenden Person und seiner Erfahrungen gezogen. Erklären wir dies an ein paar Beispielen:
\begin{enumerate}
      \item \textbf{Beispiel YouTube}: Eine Person fühlt sich gelangweilt (Langeweile als innerlicher Auslöser) also geht diese Person auf YouTube um sich ein Video anzusehen (Aktion). Durch den Auslöser wird die Person also verleitet, die Aktion durchzuführen. 
      \item \textbf{Beispiel Facebook}: Ist eine Person alleine oder fühlt sie sich alleine (innerer Auslöser), so kann sie auf Facebook gehen (Aktion) um mit Freunden oder Bekannten zu unterhalten und so dem Gefühl der Einsamkeit vorbeugen.
\end{enumerate}
Diese Beispiele lassen sich für viele soziale Media Technologien erkennen. Die Gemeinsamkeit die auch einen Teil des Erfolges dieser Technologien erklärt ist, das sie die innerlichen Auslöser der Menschen befriedigen. Viele dieser Services fangen klein an und werden über einen gewissen Zeitraum zu einer Gewohnheit des Alltages ohne die sich die meisten Menschen nicht mehr zufrieden geben würden \cite{Eyal2014}.
\\\\
Die Kombination von äußerlichen und inneren Auslösern ist nun die nächste Anforderung die aus diesem Kapitel für diese Arbeit hervorgeht. Hierbei muss sich die Gestaltung auf einfache und sehr intuitive Auslöser konzentrieren und es muss des weiteren die menschliche Seite des Dokumentationprozesses untersucht werden. Als eine generelle Idee kann der innerliche Auslöser des bekannten \enquote{AHA-Effekts} genutzt werden. Die Gestaltung der Trigger wird in den folgenden Kapiteln in Zusammenhang mit den passenden Gamification Elementen durchgenommen.  

\subsection{Anforderungsliste für den Gamification Bereich}
Nun haben wir uns mit den schwierigkeiten des Dokumentationsprozesses beschäftigt und haben in den Vorangegangenen Kapiteln Ansätze und Anforderungen gestellt die an dieser Stelle nochmal zusammengefasst werden sollen.
\begin{description}
   \item[Anforderungsliste:]~\par
   \begin{itemize}
      \item \textbf{Motivation steigern durch Belohnung}: Wie wir gesehen haben ist der konkrete Nutzen einer Dokumentation nicht Anreiz genug um Mitarbeiter auf dauer zu motivieren. Um nun die Motivation am Dokumentationsprozess zu steigern muss eine entsprechende Belohnung für diesen Prozess entwickelt werden.
      \item \textbf{Fähigkeiten des Dokumentationsprozesses verbessern}: Durch die verschiedenen Fähigkeitsmerkmale wie Zeit, Verständnis und soziale Abweichung muss der Prozess entschlackt und vereinfacht werden.
      \item \textbf{Auslöser finden und entwickeln}: Die entsprechenden äußerlichen und innerlichen Auslöser müssen genauer identifiziert werden. Desweiteren müssen die Auslöser entsprechend für den Dokumentationsprozess entwickelt werden.
   \end{itemize}
\end{description}
Auf diese drei Anforderungen kann in den kommenden Kapiteln der Fokus gesetzt werden. Nun müssen die entsprechenden Gamification Elemente herausgearbeitet werden, mit denen sich diese Liste umsetzten lässt.





 























  
\newpage
\listoftables
\listoffigures
\newpage
\bibliography{library,library2}

\end{document} 