\documentclass[a4paper,12pt]{scrartcl}
\usepackage[utf8]{inputenc}
\usepackage{lmodern}
\usepackage[ngerman]{babel}
\usepackage[autostyle=true,german=quotes]{csquotes}
\usepackage[T1]{fontenc}
\usepackage{graphicx}
\usepackage{geometry}
\usepackage{setspace}
\usepackage{amsmath}
\usepackage{colortbl}
\usepackage{multirow}
\usepackage{afterpage}
\usepackage{wrapfig}
\usepackage{fancyhdr}

\usepackage[colorlinks=true,urlcolor=blue,linkcolor=black,
citecolor=black
]{hyperref}

\bibliographystyle{unsrt}

\geometry{verbose,a4paper,tmargin=30mm,bmargin=35mm,lmargin=25mm,rmargin=25mm}
\pagestyle{fancy}
\renewcommand{\sectionmark}[1]{\markboth{#1}{#1}}
\renewcommand{\subsectionmark}[1]{\markright{#1}}
\fancyhf{}
\lhead{\slshape \leftmark}
\rhead{\thepage}
\title{Deckblatt}
\author{Matthias Misior}

\begin{document}


\thispagestyle{empty}
\section*{Erklärung}
Hiermit versichere ich, dass ich die vorliegende Arbeit selbstständig verfasst und keine anderen als die angegebenen Quellen und Hilfsmittel benutzt habe sowie wörtliche und sinngemäße Zitate als solche gekennzeichnet habe. Diese Versicherung bezieht sich sowohl auf
Textinhalte sowie alle enthaltenen Abbildungen, Skizzen und Tabellen. Die Arbeit wurde in gleicher oder ähnlicher Form keiner anderen Prüfungsbehörde zur Erlangung eines akademischen Grades vorgelegt.
\\\\
Karlsruhe, den 04. April 2018
\\\\
Matthias Misior \noindent\dotfill

\newpage
\thispagestyle{empty}
\section*{Anerkennungen}
Zunächst möchte ich mich an dieser Stelle bei meinem Betreuer Maximilian Liesegang bedanken und meine Anerkennung ihm gegenüber zum Ausdruck bringen. Seine Fachkompetenz und sein Enthusiasmus haben mich über die gesamte Projektdauer begleitet und waren stets eine große Hilfe und eine wichtige Motivationsquelle für mich. 
\\\\
Des weiteren gilt mein persönlicher dank Frau Professorin Laubenheimer, dessen ständige Erreichbarkeit und schneller E-Mail Verkehr es mir erst ermöglicht haben, diese Arbeit in dieser Form rechtzeitig zum Abgabetermin fertigzustellen. 
\\\\
Zu Guter Letzt gilt mein persönlicher Dank dem gesamten esentri-Team. Das Team hat mich inspiriert, andere Wege zu gehen und ich habe dadurch viel über meine Vorgehensweise sowie über mich selbst gelernt. Die Fachkompetenz und Ratschläge des Teams waren dabei stets eine große Hilfe. Ohne die Unterstützung des esentri-Teams wäre diese Arbeit so nicht möglich gewesen.
\\\\
Ich hoffe ich konnte mit diesen einfachen Worten allen Beteiligten meine Dankbarkeit zum Ausdruck bringen. 

\newpage
\thispagestyle{empty}
\section*{Zusammenfassung}
Die esentri AG mit Sitz in der Technologieregion Karlsruhe ist ein mittelständisches Beratungs-Unternehmen mit den Schwerpunkten Digitalisierung und IT-Systemintegration. 
Das Unternehmen plant dabei seine Projekte agil, transparent und nah beim Kunden.
Um den hohen Ansprüchen dieser Kunden nun gerecht zu werden hat es sich die esentri zur Aufgabe gemacht, den internen Dokumentationsprozess innerhalb der Firma zu verbessern. 
\\\\
Diese Aufgabe erweist sich allerdings Aufgrund der in diesem Feld auftretenden Probleme als schwer greifbar. Das Schreiben von Dokumentationen wird meistens als langweilig und monoton angesehen. Hinzukommt dass das Dokumentieren nichts zum Lösungsansatz eines Projektes beiträgt. Aus diesen und weiteren Gründen wird das verfassen von Dokumentationen von vielen Entwicklern als “Zeitverschwendung” betrachtet. 
\\\\
Um diese Tätigkeit nun für die Mitarbeiter attraktiver zu gestalten und somit die Qualität der Dokumentationen zu verbessern, hat man sich dazu entschieden den Dokumentationsprozess mit Gamification aufzuwerten. Dafür soll sich in die Grundlagen des Gebietes Gamification eingearbeitet werden um zu ermitteln, ob und wie man die aus dem Gamification Bereich stammenden Elemente nutzen kann.


\section*{Abstract}
The esentri AG with their headquarters based in the technology region Karlsruhe is an medium-sized consulting-company with a focus on digitization and IT system integration. The company plans its projects agile, transparently and close to the customer. In order to meet the high demands of these customers, esentri has set itself the task of improving the internal documentation process within the company. 
\\\\
However, this task proves to be difficult to grasp due to the problems encountered in this field. Writing documentation is often considered boring and monotonous. In addition, documenting does not contribute to the solution approach of a project. For these and other reasons, writing documentation is considered a \enquote{waste of time} by many developers. 
\\\\
In order to make this activity more attractive for employees and thus improve the quality of documentation, it was decided to enhance the documentation process with gamification. For this purpose, the basics of gamification have to be studied in order to determine whether and how the elements from gamification can be used.     
\newpage
\thispagestyle{empty}
\tableofcontents
\thispagestyle{empty}
\newpage
\bibliographystyle{unsrt}


\section{Einleitung}
Dieses Kapitel dient als Einstieg in diese Arbeit und soll die Aufgaben, Ziele, Vorgehensweise und das Arbeitsumfeld genauer beschreiben. Dabei wird zwischen Aufgabenstellung und Aufgabenbearbeitung unterschieden. Die Aufgabenstellung lässt sich aus der Thesis Beschreibung ableiten und beschreibt die Aufgaben die im laufe dieser Thesis bearbeitet werden sollen. Die Aufgabenbearbeitung enthält die generelle Vorgehensweise wie diese Aufgaben umgesetzt werden sollen. Des weiteren werden die Ziele definiert die mit dieser Arbeit erreicht werden sollen. Die Arbeitsumgebung wird ebenfalls beschrieben und es wird am Ende dieses Kapitels mit einem Überblick abgeschlossen. 
\subsection{Aufgabenstellung}
Die Aufgabe die in dieser Arbeit behandelt werden soll ist, die Tätigkeit des Dokumentierens von Software (dokumentieren, Dokumentationsprozess) für die Softwareentwickler der esentri AG motivierender und belohnender zu gestalten. Hierfür soll der Dokumentationsprozess mit Spiele-Prinzipien und Spiele-Mechaniken versehen werden. Durch die Integration von Spiele-Prinzipien und Spiele-Mechaniken sollen dem Dokumentationsverfasser zusätzliche Anreize gegeben werden, gute und aussagekräftige Dokumentationen zu schreiben. 
\\\\
Das Integrieren von Spiele-Mechaniken und Spiele-Prinzipien in nicht Spiel-Kontexten, wie in diesem konkreten Fall den Dokumentationsprozess, wird als Gamification bezeichnet.
\\\\
Für die Integration müssen die verschiedensten Spiele-Prinzipien und Spiele-Mechaniken welche als Gamification Elemente zu verstehen sind, untersucht und bewertet werden. Dafür werden im Vorfeld Kriterien festgelegt anhand derer die verschiedenen Gamification Elemente analysiert werden können. Anschließend sollen die ermittelten Gamification Elemente in das Dokumentationstool der esentri AG integriert werden. 

\subsubsection{Definition Gamification}
Der Begriff Gamification wurde im Jahr 2002 von Nick Pelling geprägt, fand zu dieser Zeit aber keine Aufmerksamkeit und hatte auch mit der heutzutage geläufigen Definition wenig zu tun.
\\\\
\enquote{Gamification is the use of game mechanics and experience design to digitally engage and motivate people to achieve their goals.}\cite{gamificationDefinition}
\\\\
Erst 2009 wurde die Idee um Gamification wieder aufgegriffen und fand mit der digitalen Medienbranche auch ein vielseitiges Einsatzgebiet. Seit dieser Zeit (2009) sind mehrere verschiedene Geschäftsfelder um die Kernidee der Gamification entstanden wie z.B der Bereich der \enquote{Serious Games} oder auch das \enquote{e-learning} Umfeld. Einen detailierter Einblick in die verschiedene Bereiche der Gamification folgt in weiteren Kapiteln. 

\subsubsection{Nutzen von Gamification}
Dieses Zitat aus dem Film Mary Poppins beschreibt ziemlich präzise die Idee und den Nutzen, die hinter Gamification stecken:\\\\ 
\enquote{In every job that must be done, there is an element of fun. You find the fun, and - SNAP - the job's a game!} \footnotemark
\footnotetext{Film Mary Poppins, 1964} 
\\\\
Es geht darum, langweilige und monotone Tätigkeiten durch Gamification Elemente motivierender und interessanter zu gestalten. Dadurch soll zum einen der Anwender inspiriert werden. Zum anderen sollen die Resultate der Tätigkeit durch engagierte Anwender verbessert werden.

\subsection{Ziel dieser Arbeit} 
Ziel dieser Arbeit soll es sein, die Dokumentationsbereitschaft von Mitarbeitern der esentri AG zu erhöhen. Als Dokumentationsbereitschaft soll der Wille und die Fähigkeit von Mitarbeitern verstanden werden, gute und aussagekräftige Dokumentationen zu schreiben. Zu diesem Zweck sollen aus dem Bereich der Gamification diejenigen Elemente recherchiert werden, die diesem Ziel am förderlichsten sind. Die Gamification Elemente sollen dann wiederum in das bestehende Dokumentationstool Confluence integriert werden.
\begin{description}
   \item[Dokumentationsbereitschaft erhöhen bedeutet:]~\par
   \begin{itemize}
      \item Den \textbf{Willen/Motivation} der Mitarbeiter erhöhen, gute und aussagekräftige Dokumentationen zu schreiben.  
      \item Die \textbf{Fähigkeiten} der Mitarbeiter verbessern, gute und aussagekräftige Dokumentationen zu schreiben.
   \end{itemize}
\end{description}

\begin{description}
   \item[Durch das beeinflussen dieser beiden Punkte sollte es also möglich sein:]~\par
   \begin{itemize}
      \item Das Schreiben von Dokumentationen für den Schreiber interessanter zu gestalten.  
      \item Die Qualität der Dokumentationen zu erhöhen.
   \end{itemize}
\end{description}
Ein weiteres Ziel dieser Arbeit ist es, das Themengebiet rund um \enquote{Gamification} der esentri AG näher zu bringen. Der esentri AG ist es ein Anliegen, mehr über die potenziellen Einsatzmöglichkeiten dieses Themengebietes zu erfahren. Die Integration von Gamification Elementen in die vorhandenen Dokumentationstools der esentri AG sind somit als eine Art \enquote{Experiment} zu verstehen. Diese Arbeit soll anhand des Beispiels des Dokumentationstools aufzeigen, ob es sich rentiert Gamification Elemente in bestehende und zukünftige Projekte zu integrieren.  
\subsection{Arbeitsumgebung}
Die esentri AG mit Sitz in Ettlingen ist eine im Jahr 2002 gegründete Beratungs-Firma, die sich auf die beiden Geschäftsbereiche Digitalisierung und IT-Systemintegration spezialisiert hat. Die esentri ist ein cloudbasiertes innovatives Unternehmen das von sich selbst behauptet:
\\\\
\enquote{Unser Antrieb ist die Neugier, mit der wir nach den aktuellsten Trends Ausschau halten und mit der wir gerne auch mal über den Tellerrand schauen.} \footnotemark
\footnotetext{\url{http://www.esentri.com/unternehmen/}}
\\\\
Als solches bearbeitet die esentri neben Ihren Geschäften auch verschiedenste interne Projekte die sich mit neuen Technologien, Formaten oder Konzepten auseinandersetzen. Zu diesen Projekten zählt auch diese Arbeit. 
\\\\
Das Unternehmen verwaltet und plant seine Projekte agil und greift hierfür auf die Produktpalette von \textit{Atlassian} zurück. Zu den wichtigsten Atlassian Tools innerhalb esentri zählen unter anderem die Aufgabenmanagementsoftware \textit{Jira} und die \textit{Confluence Wiki}.

\subsubsection{Confluence}
Das Confluence Wiki wird seit 2012 bei der esentri eingesetzt. Wichtige Eigenschaften dieses Tools beinhalten die Möglichkeit, Verknüpfungen zwischen Jira, Bitbucket und weiteren Atlassians Produkten zu erzeugen. Somit ist es einfach möglich, Aufgaben Tickets von Jira mit einer speziellen Confluence Wikipage zu verknüpfen. Das stellt einen großen Vorteil gegenüber anderen Wikitools dar.
\\\\
\enquote{Confluence ist eine kommerzielle Wiki-Software, die vom australischen Unternehmen Atlassian entwickelt und als Enterprise Wiki hauptsächlich für die Kommunikation und den Wissensaustausch in Unternehmen und Organisationen verwendet wird, aber zunehmend auch als Basis für öffentliche Wikis im Internet zum Einsatz kommt.} \footnotemark
\footnotetext{\url{https://de.wikipedia.org/wiki/Confluence_(Atlassian)}}
\\\\
Die Atlassian Produkte verfügen zusätzlich über weitere \textit{Addons} und über einen eigenen Marketplace. Somit bleiben die Produkte erweiterbar und können individuell zugeschnitten werden.

\subsection{Aufgabenbearbeitung} 
Um die besagten Ziele zu erreichen, ist es wichtig die Kernaspekte dieser Arbeit zu verstehen, sie zu betrachten und ihre zusammenhänge aufzuführen.
\begin{description}
   \item[Die drei wichtigsten Aspekte dieser Arbeit lassen sich wie folgt einteilen:]~\par
   \begin{itemize}
      \item Dokumentation
      \item Gamification
      \item Confluence
   \end{itemize}
\end{description}
Diese drei Bereiche müssen behandelt werden um die im Vorherigen Abschnitt erwähnten Ziele, wie das erhöhen der Dokumentationsbereitschaft oder das aneignen von Grundlagen Wissen im Bereich Gamification zu erreichen.
\begin{description}
   \item[Dokumentation]~\par
Unter diesem Aspekt müssen die grundlegenden Fragen beantwortet werden wie:
   \begin{itemize}
      \item Wie schreibt man eine gute und aussagekräftige Dokumentation?
      \item Was sind Merkmale einer solchen Dokumentation?
      \item Warum ist das Schreiben von Dokumentationen eine eintönige Tätigkeit?
   \end{itemize}
\end{description}
Diese Fragen müssen als aller erstes untersucht und geklärt werden um die Kriterien und Anforderungen an den Bereich der Gamification aufstellen zu können. Diese Kriterien können dann verwendet werden um den Bereich der Gamification besser einzugrenzen. Dadurch wird das Risiko minimiert, sich im Gamification Bereich zu “verlieren”. 
\\\\
\textbf{Gamification}\\
Die Kriterien und Anforderungen welche sich aus der Analyse des Dokumentationsprozesses ergeben, sollen nun genutzt werden um die entsprechenden Gamification Elemente herauszuarbeiten. Es wird also untersucht, ob es Elemente im Bereich der Gamification gibt welche auf die Anforderungen und Kriterien passen.
\\\\
Um die Gamification Elemente und die Anforderungen besser aufeinander abstimmen zu können, ist es wichtig sich ebenfalls die Grundlagen zum Thema Gamification anzueignen. Dabei ist zu beachten, dass sich die Gamification Grundlagen nicht nur in technische Lösungen einteilen lassen. Gamification behandelt die Psychologie der Motivation und das erzeugen von Verhalten. In diesem Fall wollen wir die Gamification nutzen um Mitarbeiter zu motivieren und ihr Verhalten soweit ändern, das sie bessere Dokumentationen schreiben.
\begin{description}
   \item[Also müssen folgende Forschungsfragen in diesem Abschnitt behandelt werden:]~\par
   \begin{itemize}
      \item Was ist Motivation?
      \item Wie kann man Motivation erzeugen?
      \item Wie kann Motivation genutzt werden?
   \end{itemize}
\end{description}
Wenn diese Fragen geklärt sind, ist es möglich Aussagen darüber zu treffen, welche Gamification Elemente sich in Verbindung mit den Anforderungen am besten eignen würden um die Dokumentationsbereitschaft zu erhöhen.
\\\\
\textbf{Confluence}\\
Die identifizierten Gamification Elemente müssen zu guter Letzt noch in das firmeninterne Dokumentationstool \enquote{Confluence} integriert werden. Die benötigten Kenntnisse wie Confluence zu konfigurieren ist, welche Möglichkeiten der Konfiguration Confluence zu Verfügung stellt und wie aufwendig die Integration der gewählten Gamification Elemente wird, fallen in diesen Teil der Arbeit. 
\\\\
Die Confluence Software ist ein wichtiger Baustein der esentri AG und kann deswegen nicht ohne größeren Aufwand ausgetauscht werden. Deswegen ist es wichtig die Grenzen der Software zu ermitteln. Dadurch ist die Auswahl der Gamification Elemente zwar eingeschränkt, da lediglich Elemente verwendet werden dürfen die sich mit dieser Software auch umsetzen lassen. Trotzdem gibt es Punkte die dafür sprechen Confluence als Dokumentationstool für diese Arbeit zu wählen. So kann der Atlassian Marketplace z.B nach vorhandenen Gamification-Addons durchsucht werden die bereits Gamification Elemente beinhalten.
\\\\
\textbf{Zusammenhang von Dokumentation, Gamification und Confluence}\\
In dieser Arbeit wird versucht alle Schwerpunkte der einzelnen Themenbereiche Dokumentation, Gamification und Confluence zu betrachten und diese zusammenzuführen. Dadurch entstehen natürlich Abhängigkeiten und Zusammenhänge. So können z.B die Gamification Elemente erst ausgewählt werden nachdem die Anforderungen an die Elemente aus dem Dokumentationsprozess ermittelt wurden. Ebenfalls können diese Elemente nur dann verwendet werden wenn sie sich mit Confluence umsetzen lassen. Durch diese Abhängigkeiten aller drei Bereiche unterscheidet sich diese Arbeit von anderen Arbeiten, die eben nur eines dieser Felder als Schwerpunkt hat.
\subsection{Übersicht}
Diese Arbeit teilt sich in drei Teile auf. Im ersten Teil werden hauptsächlich die Grundlagen der wichtigen Aspekte Dokumentation, Gamification und Confluence behandelt. Es sollen Begrifflichkeiten und Definitionen dieser Bereiche aufgeführt und erläutert werden. 
\begin{description}
   \item[Aus dieser Phase werden die folgenden Teilergebnisse benötigt:]~\par
   \begin{itemize}
      \item Aus dem Bereich Dokumentation sollen die Anforderungen und Kriterien für die Gamification Elemente ermittelt werden.
      \item Die Anforderungen werden nun verwendet um aus dem Bereich der Gamification die passenden Elemente zu identifizieren.
      \item Im Confluence Teil muss geprüft werden, ob sich die identifizierten Gamification Elemente auch in Confluence umsetzen lassen. Andernfalls muss nach alternativen Lösungsansätzen geforscht werden.
   \end{itemize}
\end{description}
Als Gesamtergebniss dieser ersten Phase, sind die Gamification Elemente zu verstehen die sowohl den Anforderungen entsprechen und sich auch in Confluence Umsetzen lassen.
\\\\
Im zweiten Teil dieser Arbeit wird die technische Umsetzung beschrieben. Es wird dokumentiert, wie die Confluence Wiki konfiguriert wird und wie die aus Phase Eins erarbeiteten Gamification Elemente in die Wiki zu integrieren sind. Des weiteren sollen in dieser Phase Kennzahlen und Messwerte festgehalten werden. Diese Zahlen sollen genutzt werden um die Veränderung der Dokumentationsbereitschaft zu messen. Hierfür soll ein einfacher Vergleich von Alter-Zustand zu Neuer-Zustand gezogen werden. Dabei steht der Alter-Zustand für Confluence ohne Gamification und der Neuer-Zustand für Confluence mit Gamification.
\\\\
Im letzten Teil dieser Arbeit sollen die Ergebnisse aus dem Vergleich von Alter-Zustand zu Neuer-Zustand präsentiert und untersucht werden. Diese Bewertung soll Aufschluss darüber geben wie rentable es ist, Gamification Elemente für das konkrete Beispiel des Dokumentationsprozesses zu nutzen. Zum Schluss soll die Arbeit mit einem Fazit und einem Ausblick einblicke in zukünftige Projekte geben.

\section{Knowledge Management}
In diesem Kapitel sollen die Grundlagen aus dem Bereich Knowledge Management erläutert werden, die für diese Arbeit relevant sind. Das Thema Knowledge Management ist zu umfangreich um in seiner Gesamtheit für diese Arbeit in Betracht genommen zu werden. Deswegen soll sich in dieser Arbeit nur auf das Dokumentieren (Knowledge Capturing) als Untergruppe des Knowledge Management beschränkt werden. Hierzu sollen zunächst die in diesem Kapitel verwendeten Begrifflichkeiten definiert und erklärt werden. Im Anschluß werden die verschiedenen Punkte beschrieben was der generelle nutzen einer Dokumentation ist und was die Vorteile und Schwierigkeiten des Dokumentationsprozesses sind. Desweiteren sollen die Qualitätsmerkmale einer Dokumentation aufgeführt und anhand von Beispielen erklärt werden. Darauf folgen Gründe, warum Dokumentationen ungerne geschrieben werden.
\\\\
Um den Wissensaustausch durch Dokumentationen zu unterstützen muss zuerst der Dokumentationsprozess analysiert werden. Hierfür werden Recherchen betrieben die sich mit Knowledge Sharing, Dokumentation, internal Qualität oder ähnlichen Themen beschäftigen. Dadurch sollen Faktoren bestimmt werden, die sich dabei auf den Prozess auswirken. Diese Faktoren werden am Ende als Liste Zusammengefasst und dienen als Überleitung an den Gamification Teil. In den Folgenden Kapiteln wird dann ermittelt, welche Gamification Elemente sich nutzen lassen um die Faktoren positiv zu beeinflussen. Durch diese Beeinflussungen sollte sich dann auch der Dokumentationsprozess verbessern.
\\\\
Durch das untersuchen des Knowledge Capturing soll der Bereich aus verschiedenen Perspektiven betrachtet werden. Dadurch sollen nicht nur technische Schwierigkeiten des Dokumentationsprozesses aufgedeckt werden sondern vor allem auch die psychologischen Hintergründe, die das Dokumentieren so komplex machen.
\subsection{Definitionen und Begrifflichkeiten}
Hier sind die Definitionen und Begriffe erläutert wie wir sie für diese Arbeit verwenden.
\\\\
Ein \textit{Autor} ist der Verfasser einer Dokumentation. Der Autor wird meistens der Entwickler sein, da wir uns aber nicht nur auf Software-Dokumentation beschränken, können auch andere Personen wie Technische Schreiber oder Manager als Autoren in Betracht kommen. Ein Autor ist gleichzeitig immer ein \textit{Anwender} der Confluence Wiki und kann in diesem Zusammenhang deswegen auch als solcher genannt werden.
\\\\
Ein \textit{Leser} ist die Person (Entwickler, Stakeholder etc) welche die Dokumentation liest. Für den Leser wird die Dokumentation geschrieben. Er ist auch derjenige der versucht Information aus der Dokumentation zu gewinnen. Mehrere Leser werden in dieser Arbeit zusätzlich als \textit{Publikum} verstanden.
\\\\
Eine \textit{Dokumentation} oder \textit{Wissensbasis} (Knowledge base) eines Projektes bezieht sich auf alle menschliche- und maschinen Lesbare Medien die Teile von Informationen des Projektes enthalten. In dieser Arbeit werden beide Definitionen gleichgesetzt da sich beide Begriffe auf das vorhandensein von Informationen an einem Zentralen Ort beziehen (in der Confluence Wiki). Es werden keine Unterscheidungen zwischen den verschiedenen Arten von Dokumentationen getroffen wie Software-Dokumentation oder Architektur-Dokumentation. Stattdessen wird der Begriff “Dokumentation” vereinfacht und verallgemeinert genutzt, da hier die Tätigkeiten, Informationen aufnehmen in den Vordergrund zu rücken.
\\\\
Als \textit{Dokumentationsprozess} (schreiben von Dokumentation, dokumentieren) wird in dieser Arbeit jener Prozess bezeichnet, der für das Sichern von Informationen verantwortlich ist. Dabei wird nicht auf die Art der Information eingegangen um den Begriff universaler in dieser Arbeit verwenden zu können. Desweiteren wird in dieser Arbeit ebenfalls nicht festgelegt, an welchem Ort die Information gespeichert wird. Da in dieser Arbeit die Confluence Wiki als referenzpunkt genutzt wird, wird in dieser Arbeit aber davon ausgegangen dass Informationen in diese Wiki eingetragen werden.
\\\\
Der Begriff \textit{technische Schulden} wurde 1992 von Ward Cunningham vorgestellt. Technische Schulden beschreiben dabei den Vorgang, in einem Projekt diejenigen Aktivitäten zu vernachlässigen welche keine neuen Funktionalität in das Projekt bringen. 
\\\\
Zu diesen reduzierten Aktivitäten zählen meistens die Dokumentation des Projektes und das Testen von Funktionen innerhalb des Projektes. Ein Entwickler der an diesem Projekt arbeitet verschuldet sich also, indem Dokumentation und das Testen übergangsweise vernachlässigt werden. Dadurch wird kurzzeitig die Entwicklung des Projektes vorangetrieben. Diese technischen Schulden müssen aber, wie alle Schulden, zu einem späteren Zeitpunkt zurückgezahlt werden. 
\\\\
Die Rückzahlung bei technischen Schulden erfolgt durch das erneute investieren von Zeit, um das Wissen zu erarbeiten, welches während dem Projekt zwar gewonnen wurde, aber nicht rechtzeitig dokumentiert und somit verloren ging. Dies erfordert zusätzliche Anstrengungen die als \textit{technische Zinsen} zu verstehen sind. Die Rückzahlung ist somit teurer als die eigentlichen Kosten für die Dokumentation da die Rückzahlung mit den Zinsen addiert wird.

\subsection{Zweck einer Wissensbasis}
Der Zweck einer Dokumentation besteht darin, dass relevante Information zu einem Thema gesammelt, gespeichert und aufbereitet werden. Diese Informationen werden aus verschiedenen Gründen erfasst und gespeichert.

\subsubsection{Spätere Verwendung der Information für Autor}
Eine Studie von Parnin im Jahr 2010 \cite{Parnin2010} über kognitive Prozesse von Entwicklern beschreibt das Problem im Bereich der Softwareentwicklung: Ein Entwickler dessen Aufgabe darin besteht Quellcode zu generieren verliert schon gleich nach der Implementierung des Codes Informationen die für Änderungen oder Erweiterungen des Codes wichtig wären. Dieses Wissen geht verloren, solange es nicht ausreichend dokumentiert wird. Nach nur wenigen Minuten Ablenkung verliert ein Entwickler bereits seine gut durchdachten Gedankengänge \cite{Graham2004}. Nach wenigen Tagen verblasen Informationen zu ausgewählten Namen und Identifizieren. Nach Wochen verliert ein Entwickler seine Erinnerung an die Mentale Repräsentation die benötigt wird um Aufgaben in bezug auf den Quellcode zu erfüllen. Es zeigt sich also, dass die menschliche Erinnerung nicht effektiv genug Informationen und Wissen abspeichern kann.

\subsubsection{Mitarbeiter schneller in das Projekt integrieren} 
Der Prozess des Onboardings (einen neuen Mitarbeiter in das Projekt einlernen) ist ein wichtiger Prozess in der IT Branche und besonders bei Beratungsfirmen. Die Fachkenntnisse die in Form von Beratern von Kunden eingekauft werden, bestimmen den Wert der Dienstleistung gegenüber dem Kunden. Mitarbeiter wechseln häufiger nach Bedarf oder Aufgrund von spezifischen Fachkenntnissen die Projekte. Dadurch entstehen zwei Probleme für das Projekt. Verlässt ein Mitarbeiter ein Projekt so verliert es effektiv an Wissen in Form des Mitarbeiters. Wird dieses wertvolle Wissen nicht rechtzeitig gesichert, entstehen finanzielle Kosten. Die Kosten beziehen sich dabei auf erneute Zeitinvestitionen die gemacht werden müssen, um das verlorene Wissens wieder aufzuarbeiten.
\\\\
Zudem sind neue Mitarbeiter mit dem Projekt und dessen Wissenbasis (Dokumentation) nicht vertraut. Das einarbeiten eines Mitarbeiters kostet Zeit und unter umständen Personal. Um das Einlernen effizienter zu gestalten, ist eine verständliche Wissenbasis unerlässlich. Sie bietet einen Startpunkt in das Projekt für den Mitarbeiter, gibt ihm grundlegende Informationen und ermöglicht ihm Fragen im Vorfeld eigenständig zu beantworten. Desto verständlicher und strukturierter die Dokumentation, desto effizienter ist der Onboarding-Prozess.   
\\\\
Hinzu kommt das neue Mitarbeiter ihre Probleme effizienter kommunizieren können, da die Kommunikation über die Dokumentation geführt werden kann. Somit können Senior Mitarbeiter (Mitarbeiter die mit den Projekt gut vertraut sind) einfach auf die Dokumentation verweisen wenn es Unklarheiten gibt. 

\subsubsection{Erweiterbarkeit des Projektes wird gesichert}
Durch eine strukturierte Wissensbasis in form einer Dokumentation können leichter Änderungen an den Projekten durchgeführt werden. Die Informationen die hierfür aus der Dokumentation genutzt werden verhindern, das Änderungen an der falschen Stelle gemacht werden. Die Informationen helfen ebenfalls zu ermitteln, welche Lösungsansätze bereits versucht worden sind und wie erfolgreich diese waren. Es fällt ebenfalls leichter zu erkennen ob die verlangten Änderungen umsetzbar sind oder nicht.

\subsubsection{Wartbarkeit der Software wird sicher gestellt}
Das Warten und Instandhalten von Software gehört immer mehr zu den Tätigkeiten eines Entwickler. Deswegen ist es auch hier wichtig, eine verständliche Wissensbasis mit entsprechenden Informationen zu haben die erklären, wie die Software oder das System arbeitet. Dadurch können Fehlerquellen eingegrenzt bzw. identifiziert werden. Fehler die häufiger auftreten, können ebenso in die Dokumentation aufgenommen werden wie grundsätzliche Einstellungen und Konfigurationen.
 
\subsubsection{Nutzen für verteilte Projekte}
Die Dokumentation dient bei verteilten Projekten ebenfalls als Kommunikationsmedium. Dies ist ein wichtiger Punkt für die esentri AG da nicht immer alle Mitglieder des Projektes Vor Ort sind und sich dadurch längere und vor allem kompliziertere Kommunikationswege auftun. Durch das Verweisen auf die Dokumentation, können umständlicher Kommunikationskanäle wie Telefongespräche oder E-mails reduziert werden oder komplett wegbleiben. Obwohl eine Dokumentation es nicht vermag komplett auf Meetings oder Besprechungen zu verzichten, helfen sie doch durch ein gemeinsames Wissensvokabular diese Meetings effizienter zu gestalten. 

\subsection{Die Hürden des Dokumentationsprozesses}
Das schreiben von Dokumentationen ist natürlich ein Aufwand so wie das implementieren von Code auch. Dieser Prozess kostet Zeit und ist aufgrund seiner Komplexität entsprechend schwer greifbar. Die Komplexität beim Dokumentationsprozess rührt daher, das verschiedene Einflüsse in eine Dokumentation miteinfließen.
\\\\
Zunächst müssen diejenigen Informationen identifiziert werden, welche in die Dokumentation aufgenommen werden sollen. Hierfür muss der Autor entscheiden, welche Informationen wichtig oder unwichtig sind. Diese Fähigkeit zu erlangen erfordert Übung und Erfahrung. Das Kategorisieren (Wichtig/Unwichtig) von Informationen ist eine komplexe Angelegenheit, da hier sowohl der Wissensstand des Autors als auch der Wissensstand des Lesers eine entscheidende Rolle spielt. Was für den einen Trivial und ersichtlich erscheint muss für den Anderen nicht unbedingt ebenfalls so sein. Ist die Wissenslücke zwischen Autor und Leser zu groß, wirkt sich das negativ auf die Qualität der Information und somit automatisch auf die Qualität der Dokumentation aus. Die Identifizierung und die Kategorisierung von Information kann bis heute noch nicht automatisiert werden da es bis jetzt noch kein System gibt das das menschliche Urteilsvermögen ersetzen kann.
\\\\
Der nächste Faktor in der Komplexität des Dokumentationsprozesses ist die Darstellung der Information. Die Information sollte leicht verständlich, kurz und knapp dargestellt sein. (KISS-Prinzip\footnotemark
). Nicht jeder Autor ist jedoch in der Lage seine einzigartigen Gedanken so zu formulieren, das Sie als verständlich angesehen werden. Informationen kurz oder knapp zu halten ist ebenfalls nicht so trivial wie man anfangs vermuten lässt. Schließlich müssen Informationen kompakt und auf den Punkt gebracht werden. Die Essenze einer Information zu sichern muss ebenfalls dem menschlichen Urteilsvermögen überlassen werden. Unterstützung durch Tools oder Systeme gibt es auch hier nicht.
\footnotetext{\url{http://www.community-of-knowledge.de/beitrag/keep-it-simple-stupid-leichtgewichtiges-wissensmanagement-mit-wikis/}}
\\\\
Die identifizierten und ausformulierten Informationen müssen anschließend in eine Form und Struktur übernommen werden, die logisch und verständlich für das entsprechende Publikum ist. 

\begin{description}
   \item[Dabei gilt es die Dokumentation so zu designen:]~\par
   \begin{itemize}
      \item Informationen sollten leicht und schnell zu erkennen sein.
      \item Informationen sollten säuberlich voneinander getrennt sein. 
      \item Informationen sollten entsprechend geordnet sein.
   \end{itemize}
\end{description}
Die Komplexität einer Dokumentation beschränkt sich aber nicht nur auf die entsprechenden Informationen. Jede Dokumentation muss je nach Publikum und Themengebiet unterschiedlich gestaltet werden. Zwar gibt es Standards und Normen, wie z.B der arc42-Standard als Dokumentationsstandard für Architekturen. Dennoch können die vielseitigen Sichten nicht alle in einen Standard vereint werden.
\\\\
Hinzu kommt dass das Schreiben von Dokumentationen ein tiefes Verständnis des Themengebietes voraussetzt über das dokumentiert wird. Im Falle der Softwareentwicklung, die hier als Beispiel verwendet werden soll, bedeutet es das der Implementierte Quellcode verstanden werden muss. Im Idealfall dokumentiert der entsprechende Entwickler selbst den Code, ist dies jedoch nicht der Fall und die Dokumentation wird von einem Dritten geschrieben, gehen schonmal Hintergrundinformationen die eventuell für das Projekt relevant sind verloren. Somit ist es also immer besser wenn der entsprechende Entwickler einer Funktion oder Komponente (oder ein andere teil des Quellcodes) dessen Dokumentation übernimmt. Dies ist aber nicht immer möglich, da die Zeit fehlt und erfahrene Entwickler mehr damit beschäftigt sind, neue Funktionalität zu implementieren und somit das dokumentieren unerfahrenen Entwicklern überlassen wird. 
\\\\ 
Ein weiteres schwerwiegendes Problem ist auch, dass die Dokumentation einer der ersten Aktivitäten ist die bei hohen Zeitdruck vernachlässigt wird. Bei Projekten mit einem straffen Zeitplan und näherrückenen Deadlines werden so technische Schulden gemacht um so den vorgegebenen Zeitplan einhalten zu können. Diese Schulden zu erfassen ist schwierig da sie stark kontextabhängig sind. Außerdem müssen ebenfalls die technischen Zinsen erfasst und berücksichtigt werden. Auch diese Angaben sind schwer zu erfassen und somit wird verhindert dass die gesamten Schulden sachgemäss zurückgezahlt werden können.

\subsection{Vorteile des Wissensaustausch}
Dennoch zeigen Studien das eine Dokumentation trotz Zeitaufwand und Anstrengungen sich für Firmen rentiert. In einem Experiment von Eirik Tryggeseth zu den Auswirkungen von Dokumentationen auf die Wartung von Software zeigt sich z.B, dass Änderungen an Software Projekten durch Dokumentationen sich effizienter umsetzen lassen. die Lösungsansätze und Änderungen waren dank Hintergrundinformationen zusätzlich qualitativ besser \cite{Tryggeseth1997}.
\\\\
Bei diesem Experiment wurden zwei Gruppen von Studenten mit Änderungen an einem Software-Projekt beauftragt. Eine Gruppe verwendete dabei eine Dokumentation, während die zweite Gruppe lediglich den Quellcode des Projektes zur Verfügung hatte. Die Gruppe ohne Dokumentation benötigte 21,5\% mehr Zeit ihre Änderungen umzusetzen als ihre Kollegen die über eine Dokumentation verfügten \cite{Tryggeseth1997}.
\\\\
Das hier erwähnte Experiment zeigt deutlich den größten und wichtigsten Nutzen einer vernünftigen Wissensbasis, das generieren von Zeiterspanissen. Dabei wird sich aber nicht auf die Entwicklungszeit bezogen. Die Entwicklungszeit eines Projektes soll sich in diesem Zusammenhang auf die gesamte Zeitspanne beziehen, welche das planen, implementieren, dokumentieren, testen und inbetriebnehmen eines entwickelten Projektes beinhaltet. Oder bei Agilen Projekten auf die Zeit aller durchgeführten Sprints. Diese Entwicklungszeit wird natürlich nicht beschleunigt. Die Zeitersparnis macht sich erst dann bemerkbar, Wenn das hier gefundene Wissen in anderen Projekten zur einer Verkürzung deren Entwicklungszeit führt. Dadurch wird das gesammelte Wissen also wiederverwendet und führt dadurch zu einer höheren Qualität von Projekten, bei denen das Wissen verwendet wird.
\\\\
Als solches kann eine Dokumentation als Investition in die Zukunft gesehen werden. Die Erfahrungen und das Wissen das bei vergangenen Projekten gewonnen wurde, führt zur besseren Qualität und von einer Senkung der Entwicklungszeit bei zukünftigen Projekten.
\\\\
Ein weiterer Vorteil einer stämmigen Wissensbasis ist das Vermarkten des generierten Wissens. Wie bereits erwähnt ist die esentri AG eine Beratungsfirma. Das Kapitel einer Beratungsfirma liegt in den Beratern/innen und deren Fachkenntnisse. Wird das Wissensmanagement ausgebaut und den Mitarbeitern leichter Zugang und ein schnellerer Austausch von Informationen ermöglicht, steigert sich auf lange Sicht gesehen nicht nur die Qualität der Beratungen sondern hebt esentri gegenüber seinen Konkurrenten ab.



  
\newpage
\listoftables
\listoffigures
\newpage
\bibliography{library,library2}

\end{document} 